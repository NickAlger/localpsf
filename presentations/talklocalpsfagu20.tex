% Tikz plot for UQ. This is based on the tikz code of
% Fabian Schuh for the tikz examples website
\documentclass[10pt,final,xcolor=dvipsnames]{beamer}
%\usepackage{etex}

\usetheme{Madrid}

\usepackage{xspace}
%\usepackage{enumitem}
\usepackage[mathscr]{euscript}
\usepackage{subfigure}
\hypersetup{colorlinks,linkcolor=,urlcolor=blue}

% \usetheme{Boadilla}
% \usecolortheme{seahorse}

% the macros, packages, etc:
\usepackage[absolute,overlay]{textpos}
\setbeamertemplate{items}[ball]
\setbeamertemplate{blocks}[rounded][shadow=true]
\setbeamertemplate{navigation symbols}{}
\newenvironment{reference}[2]{%
  \begin{textblock*}{\textwidth}(#1,#2)
  \scriptsize{\it\color{black}}}{\end{textblock*}}
\setbeamertemplate{blocks}[rounded]% [shadow=false]
\newenvironment{boxalertenv}{\begin{altenv}%
      {\usebeamertemplate{alerted text begin}\usebeamercolor[fg]{alerted text}\usebeamerfont{alerted text}\colorbox{bg}}
      {\usebeamertemplate{alerted text end}}{\color{.}}{}}{\end{altenv}}

\newcommand{\tcmag}[1]{\textcolor{magenta}{{#1}}}
\usepackage{graphicx}
\usepackage{multirow,color}
\usepackage{bbm}
\usepackage{amsfonts,amsmath,amssymb,amsbsy,amsthm}
\usepackage{movie15}
\usepackage{arydshln}

\makeatletter
\renewcommand*\env@matrix[1][*\c@MaxMatrixCols c]{%
  \hskip -\arraycolsep
  \let\@ifnextchar\new@ifnextchar
  \array{#1}}

%\usepackage{movie15}
%\usepackage{multimedia}
% custom commands (add your own macros here)

\definecolor{bronze}{rgb}{0.8, 0.5, 0.2}
\definecolor{goldenyellow}{rgb}{1.0, 0.87, 0.0}
\definecolor{amber}{rgb}{1.0, 0.75, 0.0}
\definecolor{azure}{rgb}{0.94, 1.0, 1.0}
\definecolor{bittersweet}{rgb}{1.0, 0.44, 0.37}
\definecolor{brass}{rgb}{0.71, 0.65, 0.26}
\definecolor{camel}{rgb}{0.76, 0.6, 0.42}
\setbeamercolor{block body}{bg=camel!30}
\usefonttheme[onlymath]{serif}
\usepackage{color}

\usepackage{amsmath}
\usepackage{mathtools}
\usepackage{tikz}
\usepackage{amsfonts,amsmath,amssymb,amsbsy,amsthm}
%\usecolortheme{tango}

%\usefonttheme{serif}     % Font theme: serif
%\usefonttheme[stillsansseriftext]{serif} 
%\renewcommand*\familydefault{\sfdefault}

\usepackage{pgf}
%\usepackage{pgfplots}
\usepackage{amssymb}
\usepackage{amsmath}
\usepackage{tikz}
\usepackage{verbatim}
%\usepackage{movie15}
\usepackage{tikz,pgfplots}

%\usepackage[active,tightpage]{preview}
%\PreviewEnvironment{tikzpicture}
\usetikzlibrary{arrows,automata}
\usetikzlibrary{positioning}
\usetikzlibrary{fit}

%\input{ccgodef}
\newcommand{\iparb}{\bar{\ipar}}
\newcommand{\euclidnorm}[1]{\left\| {#1} \right\|_2}
\newcommand{\incu}{\upupsilon}
\newcommand{\incp}{\uprho}
\newcommand{\QoIquad}{\QoI_\text{quad}}
\newcommand{\tcblue}[1]{\textcolor{blue}{#1}}

%%%%%%%%%%%%%%%%%%%%%%%%%%%%%%%%%%%%%%%%%%%%%%%%%%%%%%
%% this def file is based on the ccgodef
%%%%%%%%%%%%%%%%%%%%%%%%%%%%%%%%%%%%%%%%%%%%%%%%%%%%%
% packages
%\usepackage{booktabs}
%\usepackage[mathscr]{euscript}
%\usepackage{color}
%% mathdesgin or ams
%%\usepackage{amssymb,mathrsfs}
%\usepackage{graphicx}
%\usepackage{algorithmic,algorithm}
%\usepackage{tikz,pgfplots}
%\usepackage{upgreek}
%%\usepackage{showkeys,cite}
%\usepackage{multirow}
%\usepackage{yfonts}
%\usepackage{mathtools}
%\usepackage[normalem]{ulem}
%\usepackage{latexsym}
%\usepackage{url}
%\usepackage{amsmath,amssymb,amsbsy,amsfonts}
%\usepackage{enumitem}

\newcommand{\zapspace}{\topsep=0pt\partopsep=0pt\itemsep=0pt\parskip=0pt}

%% Notice macros
%\newcommand{\notice}[1]{\textcolor{Chameleon3}{#1}}
%\newcommand{\Notice}[1]{\textcolor{ScarletRed2}{#1}}
%\newcommand{\nnote}[1]{\noindent\emph{\textcolor{grass}{N: #1\,}}}
%\newcommand{\unote}[1]{\noindent\emph{\textcolor{blue}{U: #1\,}}}
%\newcommand{\onote}[1]{\noindent\emph{\textcolor{red}{O: #1\,}}}
%
%% colors
%\definecolor{utorange}{rgb}{0.8,0.33,0.}
%\definecolor{themec}{RGB}{51,108,121}
%\definecolor{darkred}{rgb}{.6,.1,.1}
%\definecolor{darkblue}{rgb}{.1,.1,.9}
%\definecolor{greenback}{rgb}{.19,.94,.13}
%\definecolor{orange}{rgb}{.76,.39,.13}
%\definecolor{grass}{rgb}{.19,.64,.13}
%\definecolor{sierp}{RGB}{209,28,209}
%\definecolor{bgorange}{rgb}{1.,.95,.78}
%\definecolor{grassgreen}{RGB}{92,135,39}
%\definecolor{thinbox}{rgb}{.7,.8,1.}

% general
\newcommand{\defeq}{\vcentcolon=}
%\newcommand{\del}[2]{\frac{\partial{#1}}{\partial{#2}}}
\renewcommand{\vec}[1]{{\mathchoice
                     {\mbox{\boldmath$\displaystyle{#1}$}}
                     {\mbox{\boldmath$\textstyle{#1}$}}
                     {\mbox{\boldmath$\scriptstyle{#1}$}}
                     {\mbox{\boldmath$\scriptscriptstyle{#1}$}}}}
\renewcommand{\ker}{\mathsf{ker}}
\newcommand{\ran}{\mathsf{range}}
\newcommand{\dom}{\mathsf{dom}}
\newcommand{\trace}{\mathsf{tr}}
\newcommand{\eps}{\varepsilon}
%\newcommand{\norm}[1]{\left\| {#1} \right\|}
\newcommand{\ip}[2]{{\left\langle {#1}, {#2} \right\rangle}}
\newcommand{\mip}[2]{\left\langle{#1}, {#2}\right\rangle_{\!\scriptscriptstyle{\text{M}}}}
\newcommand{\mat}[1]{\mathbf{{#1}}}
\newcommand\restr[2]{{
  \left.\kern-\nulldelimiterspace % automatically resize the bar with \right
  {#1}\vphantom{\big|} \right|_{#2}}}
\newcommand{\angles}[1]{\left\langle #1 \right\rangle}

% sets of numbers
\newcommand{\R}{\mathbb{R}}
\newcommand{\Z}{\mathbb{Z}}
\newcommand{\N}{\mathbb{N}}

\newcommand{\BB}{\mat{B}}
\newcommand{\Amat}{\mat{A}}
\newcommand{\VV}{\mat{V}}
\newcommand{\RR}{\mat{R}}

% commonly used mathcals
\newcommand{\A}{\mathcal{A}}
\newcommand{\B}{\mathcal{B}}
\newcommand{\C}{\mathcal{C}}
\newcommand{\D}{\mathcal{D}}
\newcommand{\E}{\mathcal{E}}
\newcommand{\F}{\mathcal{F}}
\newcommand{\G}{\mathcal{G}}
\newcommand{\J}{\mathcal{J}}
\newcommand{\U}{\mathcal{U}}
\newcommand{\Reg}{\mathcal{R}}

% shorthands
\newcommand{\bit}{\begin{itemize}}
\newcommand{\eit}{\end{itemize}}
\newcommand{\bdm} {\begin{displaymath}}
\newcommand{\edm} {\end{displaymath}}

% Probability (general)
\newcommand{\prob}{\mathsf{Prob}}
\newcommand{\borel}{\mathscr{B}}
\newcommand{\ave}{\mathsf{E}}
\newcommand{\GM}[2]{\mathcal{N}\!\left( {#1}, {#2}\right)}

% Bayesian stuff

\newcommand{\noisem}{\mu_{\text{noise}}}
\newcommand{\obs}{\vec{d}}
\newcommand{\ipar}{m}

\newcommand{\iparpr}{m_{\text{pr}}}
\newcommand{\iparref}{m_{\text{ref}}}
\newcommand{\iparmap}{m_{\scriptscriptstyle\text{MAP}}}
\newcommand{\dpar}{\vec{m}}
\newcommand{\dparmap}{\vec{m}_{\scriptscriptstyle\text{MAP}}}
%\newcommand{\priorm}{\mu_{\text{pr}}}
%\newcommand{\postm}{\mu_{\text{post}}}
\newcommand{\M}{\mat{M}}% the Mass matrix
%\newcommand{\ff}{\vec{f}}     % parameter-to-obs MAP
\newcommand{\ff}{\mathcal{F}} % parameter-to-obs MAP
\newcommand{\aff}{\tilde{\mathcal{F}}} % parameter-to-obs MAP
\newcommand{\noise}{\bs e}
\newcommand{\observables}{\bs d}
\newcommand{\data}{ \observables_{\rm{obs}} }
\newcommand{\uobs}{ \uu_{\rm{obs}} }
\newcommand{\datacomponent}[1]{ \observables_{\rm{obs}}^{#1} }
\newcommand{\FF}{{\ensuremath{\mat{F}}}}
\newcommand{\W}{{\ensuremath{\mat{W}}}}
\newcommand{\I}{{\ensuremath{\mat{I}}}}

\newcommand{\prior} {\pi_{\mbox{\tiny prior}}}
\newcommand{\like}{\pi_{\mbox{\tiny like}}}
\newcommand{\piobs}{\pi_{\mbox{\tiny obs}}}
\newcommand{\post}{\pi_{\mbox{\tiny post}}}
\newcommand{\pinoise}{\pi_{\mbox{\tiny noise}}}

\newcommand{\mupost}{\mu_{\mbox{\tiny post}}}
\newcommand{\muprior}{\mu_{\mbox{\tiny prior}}}

\newcommand{\ncov} {\bs{\Gamma}_{\!\mbox{\tiny noise}} }
\newcommand{\prcov} {\bs{\Gamma}_{\!\mbox{\tiny prior}} }
\newcommand{\postcov} {\bs{\Gamma}_{\!\mbox{\tiny post}} }
\newcommand{\aFF}{{\tilde{\ensuremath{\mat{F}}}}}
\newcommand{\ncovw} {\bs{W}}
\newcommand{\prcovr} {\bs{R}}

\newcommand{\Cprior}{\mathcal{C}_{\text{\tiny{prior}}}}
\newcommand{\Cpost}{\mathcal{C}_{\text{\tiny{post}}}}

\newcommand{\map} {{m}_{\mbox{\tiny MAP}} }
\newcommand{\dmap} {{\vec{m}}_{\mbox{\tiny MAP}} }
\newcommand{\mpr} {{\vec{m}}_{\mbox{\tiny pr}} }
\newcommand{\dbeta}{\vec{\beta}}
\newcommand{\dn}{\vec{n}}



\newcommand{\Hpost}  { \bs H }
\newcommand{\Hmisfit}{ \mat{H}_{\mbox{\tiny misfit}} }
\newcommand{\tHmisfit}{\tilde{\mat{H}}_{\mbox{\tiny misfit}} }
\newcommand{\HT}{\matrix{\tilde{H}}_{\mbox{\tiny misfit}}}
\newcommand{\HTinf}{\tilde{H}_{\text{misfit}}}
\newcommand{\Vr} {\bs V_r}
\newcommand{\Dr} {\bs D_r}
\newcommand{\Ir} {\bs I_r}

%\newcommand{\Ge}{ \ensuremath{\matrix{G}}_{\!\mbox{\tiny e}}}
%\newcommand{\Op}{ \ensuremath{\matrix{O}}_{\!\mbox{\tiny p}}}
\newcommand{\Ge}{ \mathcal{G}_{\!\mbox{\tiny e}}}
\newcommand{\Op}{ \mathcal{O}_{\!\mbox{\tiny p}}}
\newcommand{\HH}{ \ensuremath{\matrix{H}} }
\newcommand{\SI}{ \ensuremath{{\Sigma}} }
\newcommand{\PP}{ \ensuremath{\matrix{\Psi}} }
%\newcommand{\WW}{ \ensuremath{\matrix{W}} }
\newcommand{\WW}{ \mathcal{W}}

% Macros for three different forms of adjoint operator
\newcommand{\adjMacroMM}[1]{{#1}^*}
\newcommand{\adjMacroME}[1]{{#1}^\natural}
\newcommand{\adjMacroEM}[1]{{#1}^\diamond}
\newcommand{\adjMacroEMINV}[1]{{#1}^{-\diamond}}
\newcommand{\Aadj}{\adjMacroMM{\A}}
\newcommand{\Badj}{\adjMacroMM{\BB}}
\newcommand{\iFadj}{\adjMacroME{\iF}}
\newcommand{\Fadj}{\adjMacroME{\F}}
\newcommand{\Vadj}{\adjMacroEM{\VV}}
\newcommand{\Ladj}{\adjMacroEM{\L}}

\def\bfG{\mbox{\boldmath$G$}}
\newcommand{\mc}[1]{\mathcal{#1}}

%%%%%%%%%%%%%%%%%%%%%%%%%%%%%%%%%%%%
% custom commands to the sc14 paper
%%%%%%%%%%%%%%%%%%%%%%%%%%%%%%%%%%%%
\newcommand{\SNR}{\ensuremath{\text{SNR}}}

\newcommand{\gbf}[1]{\boldsymbol{#1}}
\newcommand{\bs}[1]{\ensuremath{\boldsymbol{#1}}}

\newcommand{\edot}{\dot{\gbf{\varepsilon}}}
\newcommand{\edotsec}{\edot_\mathrm{II}}
\newcommand{\secinve}{\edot_\mathrm{II}}

\DeclareMathOperator{\sym}{sym}
%\DeclareMathOperator{\trace}{tr}
%\DeclareMathOperator{\diag}{diag}

\newcommand{\Div}{\mbox{div}}
\newcommand{\Grad}{{\bf \nabla}}
\newcommand{\grad}{\nabla}

\newcommand{\pt}{\tilde{p}}
\newcommand{\vt}{\tilde{\vv}}
\newcommand{\qt}{\tilde{q}}
\newcommand{\betat}{\tilde{\beta}}

\newcommand{\uh}{\hat{\uu}}
\newcommand{\ph}{\hat{p}}
\newcommand{\vh}{\hat{\vv}}
\newcommand{\qh}{\hat{q}}
\newcommand{\betah}{\hat{\beta}}

\renewcommand{\vec}[1]{\gbf{#1}}
\newcommand{\cvec}[1]{{#1}}

\newcommand{\uu}{\ensuremath{\vec{u}}}
\newcommand{\vv}{\ensuremath{\vec{v}}}

\definecolor{darkblue}{rgb}{.1,.1,.9}
\definecolor{utorange}{rgb}{0.8,0.33,0.}
% this are only needed for presentations
\newcommand{\tcb}[1]{\textcolor{blue}{{#1}}}
\newcommand{\tcdb}[1]{\textcolor{utorange}{{#1}}}
\newcommand{\tcr}[1]{\ensuremath{\textcolor{red}{{#1}}}}
\newcommand{\tcdr}[1]{\textcolor{darkred}{{#1}}}
\newcommand{\tcm}[1]{\textcolor{ForestGreen}{{#1}}}
\newcommand{\tcmag}[1]{\textcolor{magenta}{{#1}}}
\newcommand{\tco}[1]{\textcolor{VioletRed}{{#1}}}
\newcommand{\tcc}[1]{\textcolor{cyan}{{#1}}}
\newcommand{\dg}[1]{\textcolor{black!65!green}{{#1}}}
\newcommand{\tcg}[1]{\textcolor{blue!70!black!30!green}{{#1}}}
\newcommand{\tcdg}[1]{\textcolor{blue!20!black!50!green}{{#1}}}
\newcommand{\tcch}[1]{\textcolor{cyan}{\hat{{#1}}}}
\newcommand{\tcmhb}[1]{\textcolor{magenta}{\hat{\gbf{{#1}}}}}
\newcommand{\tcmh}[1]{\textcolor{magenta}{\hat{{#1}}}}
\newcommand{\tcbb}[1]{\textcolor{blue}{\gbf{{#1}}}}
\newcommand{\tcgb}[1]{\textcolor{grass}{\gbf{{#1}}}}
\newcommand{\tcct}[1]{\textcolor{cyan}{\tilde{{#1}}}}
\newcommand{\tcmtb}[1]{\textcolor{magenta}{\tilde{\gbf{{#1}}}}}
\newcommand{\tcmt}[1]{\textcolor{magenta}{\tilde{{#1}}}}
\newcommand{\tcotb}[1]{\textcolor{utorange}{\tilde{\gbf{{#1}}}}}
\newcommand{\tcot}[1]{\textcolor{utorange}{\tilde{{#1}}}}
\newcommand{\tcohb}[1]{\textcolor{utorange}{\hat{\gbf{{#1}}}}}
\newcommand{\tcoh}[1]{\textcolor{utorange}{\hat{{#1}}}}

\renewcommand{\matrix}[1] {\ensuremath{\boldsymbol{#1}}}

\newcommand{\bndFS}{\Gamma_{\!\mbox{\scriptsize $FS$}}}
\newcommand{\bndB}{\Gamma_{\!\mbox{\scriptsize $B$}}}
\newcommand{\bndT}{\Gamma_{\!\mbox{\scriptsize $T$}}}

\usepackage{xspace}
\newcommand{\software}[1]{\texttt{#1}\xspace}
\newcommand{\hip}{\texttt{hIPPYlib}\xspace}
\newcommand{\bhip}{\texttt{{\bf hIPPYlib}}\xspace}
\newcommand{\Fe}{\texttt{FEniCS}\xspace}
\newcommand{\pet}{\texttt{PETSc}\xspace}


\newcommand{\del}{\partial}
%\newcommand{\D}{\mathcal{D}}
\newcommand{\Ns}{{N_s}}
\newcommand{\Ntau}{{N_{\tau}}}
\newcommand{\iFF}{\mathcal{F}}
%\newcommand{\priorm}{\muprior}
\newcommand{\Acal}{\mc{A}}
%\newcommand{\postm}{\mupost}
\newcommand{\iparpost}{\ipar_\text{post}}
\newcommand{\dd}{\vec{\bar{d}}}
\newcommand{\obsop}{\mathcal{B}}

\newcommand{\GN}{\ensuremath{\Gamma_{\!\!N}}}
\newcommand{\Exp}[1]{e^{{#1}}}
\newcommand{\GD}{\ensuremath{\Gamma_{\!\!D}}}
\newcommand{\Vg}{\mathscr{V}_{\!g}}
\newcommand{\V}{\mathscr{V}_{\!\scriptscriptstyle{0}}}
\newcommand{\eip}[2]{\left\langle{#1}, {#2}\right\rangle_{\!{\R^q}}}
\newcommand{\Wn}{\mat{W}_{\!\sigma}}
\newcommand{\cip}[2]{\left\langle{#1}, {#2}\right\rangle_{\!\CM}}
\newcommand{\CM}{\mathscr{E}}
\newcommand{\LI}{\mathscr{L}}
\renewcommand{\H}{\mathcal{H}}
\newcommand{\Nd}{{n_{\text{d}}}}
\newcommand{\Ntr}{{n_{\text{tr}}}}
\newcommand{\ipart}{m_{\scriptscriptstyle\text{true}}}
\newcommand{\ut}[1]{\ensuremath{\tilde{#1}}}

\newcommand{\LRp}[1]{\left( #1 \right)}
\newcommand{\LRs}[1]{\left[ #1 \right]}
\newcommand{\LRa}[1]{\left< #1 \right>}
\newcommand{\LRc}[1]{\left\{ #1 \right\}}

\newcommand{\xx}{\ensuremath{\boldsymbol x}}
\newcommand{\rr}{\ensuremath{\boldsymbol r}}
\newcommand{\e}{\ensuremath{\boldsymbol e}}
\newcommand{\x}{\xx}
\newcommand{\y}{\ensuremath{\boldsymbol y}}
\newcommand{\z}{\ensuremath{\boldsymbol z}}
\renewcommand{\r}{\ensuremath{\boldsymbol r}}
\newcommand{\s}{\ensuremath{\boldsymbol s}}

\newcommand{\hilb}{\mathscr{H}}
\newcommand{\half} {\ensuremath{\frac{1}{2}}}
\newcommand{\nor}[1]{\left\| #1 \right\|}
\newcommand{\equaldef}{{:=}}
\newcommand{\iFFadj}{\adjMacroMM{\iFF}}

\newcommand{\istate}{u}
\newcommand{\iparspace}{\mathcal{M}}
\newcommand{\iadj}{p}
\DeclareMathOperator*{\argmin}{argmin} 
\DeclareMathOperator{\diag}{diag} 


\newcommand{\err}{\vec{\eta}}
\newcommand{\beps}{\bs{\eps}}
\newcommand{\bbeps}{\bar{\bs{\eps}}}
\newcommand{\bnu}{\bs{\nu}}
\newcommand{\ncovbeps} {\bs{\Gamma}_{\!\mbox{\tiny \beps}} }
\newcommand{\ncovbnu} {\bs{\Gamma}_{\!\mbox{\tiny \bnu}} }
\newcommand{\norm}[2][]{\left\Vert #2\right\Vert_{#1}}
\newcommand{\dbetamap}{\vec{\beta}_{\scriptscriptstyle\text{MAP}}}
\newcommand{\betamap}{{\beta}_{\scriptscriptstyle\text{MAP}}}
\newcommand{\smallB}{\scriptscriptstyle\text{B}}
\definecolor{lightblue}{cmyk}{0.88,0.44,0,0.07}
\newcommand{\betapr}{\beta_{\text{pr}}}
\newcommand{\ntrue}{a_{\scriptscriptstyle\text{true}}}
\newcommand{\NoiseForcing}{\xi}
\newcommand{\ProbMeasClass}{\mathcal{P}}
\newcommand{\ProbMeas}{\mathbbm{P}}
\newcommand{\SigmaAlg}{F}

\newcommand{\Hmatrix}{\bs{H}}

\tikzset{
    state/.style={
           rectangle,
           rounded corners,
           draw=black, very thick,
           minimum height=3em,
           inner sep=1pt,
           text centered,
           },
}

%
\newcommand{\zapspace}{\topsep=0pt\partopsep=0pt\itemsep=0pt\parskip=0pt}

%
\newcommand{\notice}[1]{\textcolor{Chameleon3}{#1}}
\newcommand{\Notice}[1]{\textcolor{ScarletRed2}{#1}}
\newcommand{\nnote}[1]{\noindent\emph{\textcolor{grass}{N: #1\,}}}
\newcommand{\emil}[1]{\noindent\emph{\textcolor{blue}{Emil: #1\,}}}
\newcommand{\julie}[1]{\noindent\emph{\textcolor{red}{Julie: #1\,}}}
\newcommand{\unote}[1]{\noindent\emph{\textcolor{blue}{U: #1\,}}}
\newcommand{\onote}[1]{\noindent\emph{\textcolor{red}{O: #1\,}}}
\definecolor{mygreen}{rgb}{0.5,0.05,0.15}
\newcommand{\cosmin}[1]{\noindent\emph{\textcolor{mygreen}{#1\,}}}

%
\definecolor{utorange}{rgb}{0.8,0.33,0.}
\definecolor{themec}{RGB}{51,108,121}
\definecolor{darkred}{rgb}{.6,.1,.1}
\definecolor{darkblue}{rgb}{.1,.1,.9}
\definecolor{greenback}{rgb}{.19,.94,.13}
\definecolor{orange}{rgb}{.76,.39,.13}
\definecolor{grass}{rgb}{.19,.64,.13}
\definecolor{sierp}{RGB}{209,28,209}
\definecolor{bgorange}{rgb}{1.,.95,.78}
\definecolor{grassgreen}{RGB}{92,135,39}
\definecolor{thinbox}{rgb}{.7,.8,1.}

%
\newcommand{\defeq}{\vcentcolon=}
%
\renewcommand{\vec}[1]{{\mathchoice
                     {\mbox{\boldmath$\displaystyle{#1}$}}
                     {\mbox{\boldmath$\textstyle{#1}$}}
                     {\mbox{\boldmath$\scriptstyle{#1}$}}
                     {\mbox{\boldmath$\scriptscriptstyle{#1}$}}}}
\renewcommand{\ker}{\mathsf{ker}}
\newcommand{\ran}{\mathsf{range}}
\newcommand{\dom}{\mathsf{dom}}
\newcommand{\trace}{\mathsf{tr}}
\newcommand{\eps}{\varepsilon}
\newcommand{\norm}[1]{\left\| {#1} \right\|}
\newcommand{\ip}[2]{{\left\langle {#1}, {#2} \right\rangle}}
\newcommand{\mip}[2]{\left\langle{#1}, {#2}\right\rangle_{\!\scriptscriptstyle{\text{M}}}}
\newcommand{\mat}[1]{\mathbf{{#1}}}
\newcommand\restr[2]{{
  \left.\kern-\nulldelimiterspace %
  {#1}\vphantom{\big|} \right|_{#2}}}
\newcommand{\angles}[1]{\left\langle #1 \right\rangle}

%
\newcommand{\R}{\mathbb{R}}
\newcommand{\Z}{\mathbb{Z}}
\newcommand{\N}{\mathbb{N}}

\newcommand{\BB}{\mat{B}}
\newcommand{\Amat}{\mat{A}}
\newcommand{\VV}{\mat{V}}
\newcommand{\RR}{\mat{R}}

%
\newcommand{\A}{\mathcal{A}}
\newcommand{\B}{\mathcal{B}}
\newcommand{\C}{\mathcal{C}}
\newcommand{\D}{\mathcal{D}}
\newcommand{\E}{\mathcal{E}}
\newcommand{\F}{\mathcal{F}}
\newcommand{\G}{\mathcal{G}}
\newcommand{\J}{\mathcal{J}}
\newcommand{\U}{\mathcal{U}}
\newcommand{\Reg}{\mathcal{R}}

%
\newcommand{\bit}{\begin{itemize}}
\newcommand{\eit}{\end{itemize}}
\newcommand{\bdm} {\begin{displaymath}}
\newcommand{\edm} {\end{displaymath}}

%
\newcommand{\prob}{\mathsf{Prob}}
\newcommand{\borel}{\mathscr{B}}
\newcommand{\ProbMeasClass}{\mathcal{P}}
\newcommand{\ProbMeas}{\mathbbm{P}}
\newcommand{\SigmaAlg}{F}
\newcommand{\ave}{\mathsf{E}}
\newcommand{\GM}[2]{\mathcal{N}\!\left( {#1}, {#2}\right)}

%
\newcommand{\NoiseForcing}{\xi}
\newcommand{\noisem}{\mu_{\text{noise}}}
\newcommand{\obs}{\F}%
\newcommand{\dobs}{{\vec{d}}}
\newcommand{\obsT}{{\vec{d}_{\scriptscriptstyle\text{obs}}}}
\newcommand{\ipar}{m}
\newcommand{\iparT}{m_{\mathrm T}}
\newcommand{\iparpr}{m_{\text{prior}}}
\newcommand{\iparmap}{m_{\scriptscriptstyle\text{MAP}}}
\newcommand{\dpar}{\vec{m}}
\newcommand{\dparmap}{\vec{m}_{\scriptscriptstyle\text{MAP}}}
%
%
\newcommand{\M}{\mat{M}}%
%
\newcommand{\ff}{\mathcal{F}} %
\newcommand{\noise}{\bs e}
\newcommand{\observables}{\bs d}
\newcommand{\data}{ \observables_{\rm{obs}} }
\newcommand{\uobs}{ \uu_{\rm{obs}} }
\newcommand{\datacomponent}[1]{ \observables_{\rm{obs}}^{#1} }
\newcommand{\FF}{{\ensuremath{\mat{F}}}}
\newcommand{\W}{{\ensuremath{\mat{W}}}}
\newcommand{\I}{{\ensuremath{\mat{I}}}}
\newcommand{\ScoreFcn}{S}
\newcommand{\Nobs}{M}
\newcommand{\Expectation}{\mathbb{E}}


\newcommand{\prior} {\pi_{\mbox{\tiny prior}}}
\newcommand{\like}{\pi_{\mbox{\tiny like}}}
\newcommand{\piobs}{\pi_{\mbox{\tiny obs}}}
\newcommand{\pinoise}{\pi_{\mbox{\tiny $\xi$}}}
\newcommand{\post}{\pi_{\mbox{\tiny post}}}

\newcommand{\mupost}{\mu_{\mbox{\tiny post}}}
\newcommand{\muprior}{\mu_{\mbox{\tiny prior}}}

\newcommand{\ncov} {\bs{\Gamma}_{\!\mbox{\tiny noise}} }
\newcommand{\prcov} {\bs{\Gamma}_{\!\mbox{\tiny prior}} }
\newcommand{\postcov} {\bs{\Gamma}_{\!\mbox{\tiny post}} }

\newcommand{\Cprior}{\mathcal{C}_{\text{\tiny{prior}}}}
\newcommand{\Cpost}{\mathcal{C}_{\text{\tiny{post}}}}

\newcommand{\map} {{m}_{\mbox{\tiny MAP}} }
\newcommand{\dmap} {{\vec{m}}_{\mbox{\tiny MAP}} }
\newcommand{\mpr} {{\vec{m}}_{\mbox{\tiny pr}} }
\newcommand{\dbeta}{\vec{m}}

\newcommand{\Hpost}  { \bs H }
\newcommand{\Hmisfit}{ \mat{H}_{\mbox{\tiny misfit}} }
\newcommand{\tHmisfit}{\tilde{\mat{H}}_{\mbox{\tiny misfit}} }
\newcommand{\HT}{\matrix{\tilde{H}}_{\mbox{\tiny misfit}}}
\newcommand{\HTinf}{\tilde{H}_{\text{misfit}}}
\newcommand{\Vr} {\bs V_r}
\newcommand{\Dr} {\bs D_r}
\newcommand{\Ir} {\bs I_r}

%
%
\newcommand{\Ge}{ \mathcal{G}_{\!\mbox{\tiny e}}}
\newcommand{\Op}{ \mathcal{O}_{\!\mbox{\tiny p}}}
\newcommand{\HH}{ \ensuremath{\matrix{H}} }
\newcommand{\SI}{ \ensuremath{{\Sigma}} }
\newcommand{\PP}{ \ensuremath{\matrix{\Psi}} }
%
\newcommand{\WW}{ \mathcal{W}}

%
\newcommand{\adjMacroMM}[1]{{#1}^*}
\newcommand{\adjMacroME}[1]{{#1}^\natural}
\newcommand{\adjMacroEM}[1]{{#1}^\diamond}
\newcommand{\adjMacroEMINV}[1]{{#1}^{-\diamond}}
\newcommand{\Aadj}{\adjMacroMM{\A}}
\newcommand{\Badj}{\adjMacroMM{\BB}}
\newcommand{\iFadj}{\adjMacroME{\iF}}
\newcommand{\Fadj}{\adjMacroME{\F}}
\newcommand{\Vadj}{\adjMacroEM{\VV}}
\newcommand{\Ladj}{\adjMacroEM{\L}}

\def\bfG{\mbox{\boldmath$G$}}
\newcommand{\mc}[1]{\mathcal{#1}}

%
%
%
\newcommand{\SNR}{\ensuremath{\text{SNR}}}

\newcommand{\gbf}[1]{\boldsymbol{#1}}
\newcommand{\bs}[1]{\ensuremath{\boldsymbol{#1}}}

\newcommand{\edot}{\dot{\gbf{\varepsilon}}}
\newcommand{\edotsec}{\edot_\mathrm{II}}
\newcommand{\secinve}{\edot_\mathrm{II}}

\DeclareMathOperator{\sym}{sym}
%
%

\newcommand{\Div}{\mbox{div}}
\newcommand{\Grad}{{\bf \nabla}}

\newcommand{\pt}{\tilde{p}}
\newcommand{\vt}{\tilde{\vv}}
\newcommand{\qt}{\tilde{q}}
\newcommand{\betat}{\tilde{\beta}}

\newcommand{\uh}{\hat{\uu}}
\newcommand{\ph}{\hat{p}}
\newcommand{\vh}{\hat{\vv}}
\newcommand{\qh}{\hat{q}}
\newcommand{\betah}{\hat{\beta}}

\newcommand{\permeab}{\mathrm{m}}

\renewcommand{\vec}[1]{\gbf{#1}}
\newcommand{\cvec}[1]{{#1}}

\newcommand{\uu}{\ensuremath{\vec{u}}}
\newcommand{\vv}{\ensuremath{\vec{v}}}

\newcommand{\tcb}[1]{\textcolor{blue}{{#1}}}
\newcommand{\tcdb}[1]{\textcolor{darkblue}{{#1}}}
\newcommand{\tcr}[1]{\ensuremath{\textcolor{red}{{#1}}}}
\newcommand{\tcdr}[1]{\textcolor{darkred}{{#1}}}
\newcommand{\tcm}[1]{\textcolor{ForestGreen}{{#1}}}
\newcommand{\tco}[1]{\textcolor{VioletRed}{{#1}}}
\newcommand{\tcc}[1]{\textcolor{cyan}{{#1}}}
\newcommand{\dg}[1]{\textcolor{black!65!green}{{#1}}}
\newcommand{\tcg}[1]{\textcolor{blue!70!black!30!green}{{#1}}}
\newcommand{\tcdg}[1]{\textcolor{blue!20!black!50!green}{{#1}}}
\newcommand{\tcch}[1]{\textcolor{cyan}{\hat{{#1}}}}
\newcommand{\tcmhb}[1]{\textcolor{magenta}{\hat{\gbf{{#1}}}}}
\newcommand{\tcmh}[1]{\textcolor{magenta}{\hat{{#1}}}}
\newcommand{\tcbb}[1]{\textcolor{blue}{\gbf{{#1}}}}
\newcommand{\tcgb}[1]{\textcolor{grass}{\gbf{{#1}}}}
\newcommand{\tcct}[1]{\textcolor{cyan}{\tilde{{#1}}}}
\newcommand{\tcmtb}[1]{\textcolor{magenta}{\tilde{\gbf{{#1}}}}}
\newcommand{\tcmt}[1]{\textcolor{magenta}{\tilde{{#1}}}}
\newcommand{\tcotb}[1]{\textcolor{utorange}{\tilde{\gbf{{#1}}}}}
\newcommand{\tcot}[1]{\textcolor{utorange}{\tilde{{#1}}}}
\newcommand{\tcohb}[1]{\textcolor{utorange}{\hat{\gbf{{#1}}}}}
\newcommand{\tcoh}[1]{\textcolor{utorange}{\hat{{#1}}}}



\renewcommand{\matrix}[1] {\ensuremath{\boldsymbol{#1}}}

\newcommand{\bndFS}{\Gamma_{\!\mbox{\scriptsize $FS$}}}
\newcommand{\bndB}{\Gamma_{\!\mbox{\scriptsize $B$}}}
\newcommand{\bndT}{\Gamma_{\!\mbox{\scriptsize $T$}}}

\usepackage{xspace}
\newcommand{\software}[1]{\texttt{#1}\xspace}
\newcommand{\hip}{\texttt{hIPPYlib}\xspace}
\newcommand{\bhip}{\texttt{{\bf hIPPYlib}}\xspace}
\newcommand{\Fe}{\texttt{FEniCS}\xspace}
\newcommand{\pet}{\texttt{PETSc}\xspace}


\newcommand{\del}{\partial}
%
\newcommand{\Ns}{{N_s}}
\newcommand{\Ntau}{{N_{\tau}}}
\newcommand{\iFF}{\mathcal{F}}
%
\newcommand{\Acal}{\mc{A}}
%
\newcommand{\iparpost}{\ipar_\text{post}}
\newcommand{\dd}{\vec{\bar{d}}}
\newcommand{\obsop}{\mathcal{B}}

\newcommand{\GN}{\ensuremath{\Gamma_{\!\!N}}}
\newcommand{\Exp}[1]{e^{{#1}}}
\newcommand{\GD}{\ensuremath{\Gamma_{\!\!D}}}
\newcommand{\Vg}{\mathscr{V}_{\!g}}
\newcommand{\V}{\mathscr{V}_{\!\scriptscriptstyle{0}}}
\newcommand{\eip}[2]{\left\langle{#1}, {#2}\right\rangle_{\!{\R^q}}}
\newcommand{\Wn}{\mat{W}_{\!\sigma}}
\newcommand{\cip}[2]{\left\langle{#1}, {#2}\right\rangle_{\!\CM}}
\newcommand{\CM}{\mathscr{E}}
\newcommand{\LI}{\mathscr{L}}
\renewcommand{\H}{\mathcal{H}}
\newcommand{\Nd}{{n_{\text{d}}}}
\newcommand{\Ntr}{{n_{\text{tr}}}}
\newcommand{\ipart}{m_{\scriptscriptstyle\text{true}}}
\newcommand{\epsobs}{\varepsilon_{\scriptscriptstyle\text{obs}}}
\newcommand{\grad}{\nabla}
\newcommand{\ut}[1]{\ensuremath{\tilde{#1}}}

\newcommand{\LRp}[1]{\left( #1 \right)}
\newcommand{\LRs}[1]{\left[ #1 \right]}
\newcommand{\LRa}[1]{\left< #1 \right>}
\newcommand{\LRc}[1]{\left\{ #1 \right\}}

\newcommand{\xx}{\ensuremath{\boldsymbol x}}
\newcommand{\rr}{\ensuremath{\boldsymbol r}}
\newcommand{\e}{\ensuremath{\boldsymbol e}}
\newcommand{\x}{\xx}
\newcommand{\y}{\ensuremath{\boldsymbol y}}
\newcommand{\z}{\ensuremath{\boldsymbol z}}
\renewcommand{\r}{\ensuremath{\boldsymbol r}}
\newcommand{\s}{\ensuremath{\boldsymbol s}}

\newcommand{\hilb}{\mathscr{H}}
\newcommand{\half} {\ensuremath{\frac{1}{2}}}
\newcommand{\nor}[1]{\left\| #1 \right\|}
\newcommand{\equaldef}{{:=}}
\newcommand{\iFFadj}{\adjMacroMM{\iFF}}

\newcommand{\NEns}{{N_s}}
\newcommand{\ES}{{\mathrm{ES}}}

\newcommand{\CovFcn}{\rm{k}}
\newcommand{\dt}{{\Delta t}}
\newcommand{\tidx}{k}

\newcommand{\XGen}{{X_{\rm Gen}}}
\newcommand{\YGen}{{Y_{\rm Gen}}}
\newcommand{\YNet}{{Y_{\rm Net}}}

\newcommand{\di}{\mathrm{d}}


\newcommand{\priorcov}{\mat{\Gamma}{\scriptscriptstyle\text{prior}}}
\newcommand{\priordensity}{\pi_{\scriptscriptstyle\text{prior}}}

\newcommand{\istate}{u}
\newcommand{\iparspace}{\mathcal{M}}
\newcommand{\iadj}{p}
\DeclareMathOperator*{\argmin}{argmin} 
\DeclareMathOperator{\diag}{diag} 
\newcommand{\iparref}{m_{\scriptscriptstyle\text{ref}}}
\newcommand{\Hmatrix}{\bs{H}}

%\DeclareMathOperator{\diag}{diag}
%\newcommand{\Hmatrix}{\bs{H}}
% argument #1: any options
\newenvironment{customlegend}[1][]{%
    \begingroup
    % inits/clears the lists (which might be populated from previous
    % axes):
    \csname pgfplots@init@cleared@structures\endcsname
    \pgfplotsset{#1}%
}{%
    % draws the legend:
    \csname pgfplots@createlegend\endcsname
    \endgroup
}
\def\addlegendimage{\csname pgfplots@addlegendimage\endcsname}

% title, authors, etc:
\title[]{Fast high-rank Hessian approximation for Bayesian ice sheet
  inverse problems}

\author[Nick Alger]{{Nick Alger}\inst{1}\\[2ex]
  {\small \textcolor{themec}{Joint work with:}}
  \\
  {\small No\'{e}mi Petra},\inst{2}
  {\small Tucker Hartland},\inst{2}
  {\small Omar Ghattas\inst{1}}}

\institute[UCM]{%
  \inst{1}{Oden Institute\\
    The University of Texas at Austin}\\\smallskip
  \inst{2}{Applied Mathematics, School of Natural Sciences\\
    University of California, Merced}\\\smallskip
}

\date[December 14, 2020]{%
  \footnotesize
  NG003--Advances in Data Assimilation, Predictability, and Uncertainty Quantification I\\
  American Geophysical Union Conference\\
  December 14, 2020}

 \begin{document}
%%----------- titlepage ----------------------------------------------%
\begin{frame}[plain]
  \titlepage
\end{frame}
% -------------------------------------------------------------------
\begin{frame}
\frametitle{Ice sheet dynamics: forward and inverse}

\begin{block}{Balance of linear momentum, mass, and energy}
  \[
  \begin{aligned}
    % balance of linear momentum:
    - \gbf{\nabla} \cdot \left[ \eta(\theta,\gbf{u}) \, \edot
      -\gbf{I}p \right] &= \rho \gbf{g},
    &[\edot = \tfrac 12  (\gbf{\nabla u} + \gbf{\nabla  u}^T)]
    \hspace{-2em}
    \\
    % balance of mass:
    \gbf{\nabla} \cdot \gbf{u} &= 0,
    \vspace{-2em} \\
    % conservation of energy:
    \rho c \left(\frac{\partial \theta}{\partial t} + \gbf{u} \cdot \gbf{\nabla}
    \theta\right)  - \gbf{\nabla} \cdot (K \gbf{\nabla} \theta)
    &= 2 \, \eta \, \mathrm{tr}(\edot^2)
  \end{aligned}
  \]
\end{block}
\vspace{0.2cm}
\begin{minipage}{.49\columnwidth}
  \alert{Mathematical challenges:}
  \begin{itemize}
    \small
  \item highly ill-conditioned linear systems
  \item complex, high aspect ratio geometry
  \item strong nonlinearities
  \item multiphysics couplings
  \item uncertain \alert{basal boundary conditions}, topography,
    heat flux
  \item observational data: InSAR, laser altimetry, GRACE
    satellite, ice cores, radar
  \end{itemize}
\end{minipage}
\begin{minipage}{.48\columnwidth}
  \alert{Inference of basal bdry. cond.}
  \begin{itemize}
    \small
  \item critical for climate simulations % (``spin-up'')
    %\item effective friction/sliding coefficient $\beta(x)$
  \item inference of effective sliding/friction coefficient $\beta$ in
    Robin boundary condition
    \begin{equation*}
      {\bf T}(\bs \sigma {\bf n}) + \tcr{\beta (x)}{\bf T}{\bf u} = 0
    \end{equation*}
    ($\bf T$ is tangential component)  from
    surface velocity observations
  \end{itemize}
\end{minipage}
\end{frame}
%---------------------------------------------------------------------------------%
\begin{frame}
  \frametitle{Bayesian approach to inverse problems} Inverse problem:
  given (possibly noisy) data $\bs d$ and (a possibly uncertain) model
  $\bs f$, infer parameters $\bs \beta$ that characterize the model, i.e.,
  \begin{equation*}
    \bs f(\bs \beta) + \bs e = \bs d
  \end{equation*}
  Interpret $\bs \beta$, $\bs d$ as random variables;
  solution of inverse problem is the ``posterior'' probability density function
  $\pi_{\mbox{\tiny post}}(\bs \beta)$ for
  $\bs \beta$:\\
  \begin{center}
    \begin{tikzpicture}[domain=-.5:3,scale=1]
      \draw[->] (-0.2,0) -- (3.2,0) node[above] {\small $\pi_{\mbox{\tiny post}}(\bs \beta)$};
      \draw[->] (0,-0.2) -- (0,1.2) node[above] {};
      \draw[color=red, samples=200] plot (\x,{1/(0.5*sqrt(2*3.1415))*exp(-(\x-1.)*(\x-1.)/0.16/2) + 1/(1.8*sqrt(2*3.1415))*exp(-(\x-2.1)*(\x-2.1)/0.08/1) +
        1/(4*sqrt(2*3.1415))*exp(-(\x-0.5)*(\x-0.5)/0.06/1)}) node[right]{};
    \end{tikzpicture}
  \end{center}
%  \only<1>{
    \begin{block}{Remarks:}
      \begin{itemize}
      \item Bayesian framework quantifies uncertainty in the inverse
        solution, given uncertainty in the prior, the data, and the
        model.
      \item Prior incorporates known information and in infinite
        dimensions chosen to act as a regularization.
      \item Bayesian solution is probability density in as many dimensions as
        the number of parameters.
      \end{itemize}
    \end{block}
\end{frame}
%---------------------------------------------------------------------------------%
\begin{frame}
  \frametitle{Exploring the posterior}

  If the prior is Gaussian with mean $\mpr$ and covariance $\prcov$,
  and we assume additive Gaussian noise in the measurements, i.e.,
  $\bs{e} \sim \mathcal{N}(\bs{0}, \ncov )$, then we obtain for the
  posterior pdf:
  \bdm
  \post(\bs{\beta})  \propto
  \exp \left( - \tfrac{1}{2} \parallel \bs{f}(\bs{\beta}) -
  \bs{d}_{\mbox{\tiny}}
  \parallel^{2}_{\ncov^{-1}} \!
  - \tfrac{1}{2}\parallel \bs{\beta} - \bs{\beta}_{\mbox{\tiny pr}}
  \parallel^{2}_{\prcov^{-1}}
  \right)
  \edm
  \vspace{0.2cm}
  The ``maximum a posteriori'' point is
  \begin{alignat*}{2}
    \bs{\beta}_{\mbox{\tiny MAP}}& \stackrel{\mathrm{def}}{=} \arg \;
    \max_{\bs \beta} \; \pi_{\mbox{\tiny post}}(\bs{\beta})
    \\
    &= \arg \;
    \min_{\bs \beta} \; \frac{1}{2} \left\|  \bs{f}(\bs{\beta}) -\dobs
    \right\|^2_{\ncov^{-1}}
    + \frac{1}{2} \left\|{\bs{\beta} - \bs{\beta}_{\mbox{\tiny pr}}}\right\|^2_{\prcov^{-1}}
  \end{alignat*}
  $\Rightarrow$ deterministic inverse problem with appropriate weighted norms!
  \vspace{1cm}
  \begin{itemize}
  \item [] \scriptsize{Details in: J. Kaipio and E. Somersalo, Statistical and Computational Inverse Problems, 2005}
  \end{itemize}
\end{frame}
%---------------------------------------------------------------------------------%
\begin{frame}
  \frametitle{The Hessian (of the negative log posterior) plays a
    critical role in inverse problems}
  \begin{itemize}
  \item Its spectral properties characterize the degree of
    ill-posedness.
  \item The Hessian drives Newton-type optimization algorithms for
    solving the inverse problem.
  \item The inverse of the Hessian locally characterizes the uncertainty
    in the solution of the inverse problem (under the Gaussian
    assumption, it is precisely the posterior covariance matrix).
  \item \alert{Goal:} rapidly perform linear algebraic operations, i.e.,
    manipulation of the Hessian (and its square root and inverse)
    actions needed by sampling or CG solvers, hence \alert{seek
      approaches to approximate the Hessian(-applies)}.
  \item These approximations will then be used as
    \alert{pre-conditioners}, and to \alert{build MCMC proposals} based
    on local Gaussian approximations.
  \end{itemize}
\end{frame}
%---------------------------------------------------------------------------------%
\begin{frame}
\frametitle{Scalability of an ice sheet inverse solver}
\framesubtitle{Inexact Newton-CG}

  \only<1>{
\begin{table}
\begin{center}
  \small
  \label{tbl:performance}
  \begin{tabular}{|r|r|r|r|r|r|}
    \hline
        {\bf \#sdof}    &   {\bf \#pdof}   &    {\bf  \#N}   &  {\bf  \#CG}
        & {\bf  avgCG}   &  {\bf \#Stokes} \\
        \hline
        %69101 &   7485 &  36 &  2534 &    70 & 1609 &     6713 \\
        95,796 &  10,371 &  42 &  2718 &    65         &     7031 \\
                                              % 1553 &    
        233,834 &  25,295 &  39 &  2342 &    60        &     6440 \\
                                              % 1717 &
        848,850 &  91,787 &  39 &  2577 &    66        &     6856 \\
                                              % 1663 &
        3,372,707 & 364,649 &  39 &  2211 &    57       &     6193 \\
                                              % 1732 
        %5667771 & 365709 &  43 &  2139 &    50 &     1978 &     6299\\
        22,570,303 & 1,456,225 &  40 &  1923 &    48     &     5376\\
                                              % 1490
        \hline
  \end{tabular}
\end{center}
\end{table}
  }
  \only<2>{
\begin{table}
\begin{center}
  \small
  %\caption{Scalability of inverse solver}
  \label{tbl:performance}
  \begin{tabular}{|r|r|r|r|r|r|}
    \hline
        {\bf \#sdof}    &   {\bf \#pdof}   &    {\bf  \#N}   &  {\bf  \#CG}
        & {\bf  avgCG}   &  {\bf \#Stokes} \\
        \hline
        %69101 &   7485 &  36 &  2534 &    70 & 1609 &     6713 \\
        95,796 &  10,371 &  42 &  2718 &    65         &     {\bf\alert{7031}} \\
                                              % 1553 &    
        233,834 &  25,295 &  39 &  2342 &    60        &     {\bf\alert{6440}} \\
                                              % 1717 &
        848,850 &  91,787 &  39 &  2577 &    66        &     {\bf\alert{6856}} \\
                                              % 1663 &
        3,372,707 & 364,649 &  39 &  2211 &    57       &     {\bf\alert{6193}} \\
                                              % 1732 
        %5667771 & 365709 &  43 &  2139 &    50 &     1978 &     6299\\
        22,570,303 & 1,456,225 &  40 &  1923 &    48     &     {\bf\alert{5376}}\\
                                              % 1490
        \hline
  \end{tabular}
\end{center}
\end{table}
  }
  
\small
\begin{itemize}
\item {\bf \#sdof}: number of degrees of freedom for the
  state variables;
\item {\bf \#pdof}: number of degrees of freedom for the
  inversion parameter field;
\item {\bf \#N}: number of Newton iterations;
\item {\bf \#CG}, {\bf avgCG}: total and average (per Newton
  iteration) number of CG iterations;
\item {\bf \#Stokes}: total number of
  linear(ized) Stokes solves (from forward, adjoint, and incremental
  forward and adjoint problems)
\end{itemize}

\begin{itemize}
\item [] \scriptsize{Details in: T. Isaac, N. Petra, G. Stadler, and
  O. Ghattas. {\em Scalable and efficient algorithms for the
    propagation of uncertainty from data through inference to
    prediction for large-scale problems, with application to flow of
    the Antarctic ice sheet}, Journal of Computational Physics, 296,
  348-368 (2015).}
\end{itemize}

\end{frame}
%---------------------------------------------------------------------------------%
\begin{frame}
\frametitle{MCMC sampling: stochastic Newton}
\framesubtitle{Performance results / Convergence diagnostics}
%\vspace{0.1in}
\small

\newcommand{\proposal}{\bs y}
\newcommand{\params}{\bs m}


\only<1>{
\begin{table}[t]
  \begin{center}
    %\def\firstrowcolor{}
    %\def\secondrowcolor{}
    %\only<1>{\def\firstrowcolor{\rowcolor{gray}}}
    %\only<2>{\def\secondrowcolor{\rowcolor{red}}}
    \begin{tabular}{|l|c|c|c|c|c|c|c|}
      \hline\hline
          {} & {\bfseries MPSRF} & {\bfseries IAT} & {\bfseries ESS} & {\bfseries
            MSJ} & {\bfseries ARR} & {\bfseries $\#$Stokes} & {\bfseries time (s)}\\
          \hline%\hline
          %\multicolumn{8}{|c|}{small noise case (more Gaussian) }\\
          %\hline
          %ind. sampler & 1.210 & 253 & 2075  & 1456 & 41 & 2783  & 139\\
          %frozen Hessian  & 1.001 & 6 & 84004 & 1390 & 40 & 72     & 4  \\
          %dynamic Hessian & 1.073 & 125 & 4032  & 565  & 17 & 1375 & 69\\
          %\hline
          %\multicolumn{8}{|c|}{large noise case (more non-Gaussian)}\\
          %\hline
          %SNIS  & 1.507 & 435 & 1207 & 280 & 9  & 4350 &  218\\
          %SNMAP  & 1.045 & 80 & 6563 & 190  & 6  & 960  & 48 \\
          SN & 1.348 & 600 & 875  & 64   & 2  & {\bf \alert{8400}} & 420 \\
          \hline\hline
    \end{tabular}
  \end{center}
\end{table}
}
\vspace{-0.05in}

\begin{columns}
  \begin{column}{0.45\columnwidth}
    \begin{itemize}
    \item {\bfseries MPSRF:} multivariate potential scale reduction factor
      \vspace{-0.02in}
    \item {\bfseries IAT:} integrated autocorrelation time
      \vspace{-0.02in}
    \item {\bfseries ESS:} effecitive sample size
      \vspace{-0.02in}
    \item {\bfseries MSJ:} mean squared jump distance
    \end{itemize}
  \end{column}
  \begin{column}{0.45\columnwidth}
    \begin{itemize}
    \item {\bfseries ARR:} average rejection rate
      \vspace{-0.02in}
    \item {\bfseries $\#$Stokes:} $\#$ of Stokes solves per independent sample
      \vspace{-0.02in}
    \item {\bfseries time:} time per independent sample
    \end{itemize}
  \end{column}
\end{columns}

\begin{columns}
  \hspace{0.26in}
  \begin{column}{0.95\columnwidth}
    \begin{itemize}
    \item {\bfseries Statistics:} 21 parallel chains (each 25k); $\#$
      samples: 525k; dof: 139; rank Hessian: 15
    %\item SNIS, SNMAP and SN are variants of the stochastic Newton
    %  MCMC method
    \end{itemize}
  \end{column}
  \begin{column}{0.5\columnwidth}
  \end{column}
\end{columns}

\vspace{0.05in}

%\vspace{-0.05in}
 \begin{align*}
   %\tcdb{q(\params_k,\proposal) =
     \mbox{ Proposal density: } \frac { \det \bs H^{1/2} }{(2\pi)^{n/2} }
      \exp \left( - \frac 12 \left( \proposal - \bs{\beta}_k + \bs H^{-1} \bs
      g \right) ^T \bs H \left( \proposal - \bs{\beta}_k + \bs H^{-1} \bs g \right)
      \right)
      %}
 \end{align*}
 
\begin{itemize}
\item [] \scriptsize{Details in: N. Petra, J. Martin, G. Stadler,
  O. Ghattas. {\em A computational framework for
    infinite-dimensional Bayesian inverse problems: Part
    II. Stochastic Newton MCMC with application to ice sheet inverse
    problems}, SIAM Journal on Scientific Computing, 2014}
\end{itemize}

\end{frame}

%---------------------------------------------------------------------------------%
\begin{frame}
\frametitle{Low-rank approximation of the Hessian of the likelihood}
% (Gauss-Newton)

\framesubtitle{Under the assumption of Gaussian noise and linearized
  parameter-to-observable map}

\begin{itemize}

\item Invoke low-rank approximation  given by the inverse of the Hessian:
  \begin{minipage}{0.9\textwidth}
  \begin{block}{}
    \vspace{-0.5cm} 
    \[
    \postcov = \Hmatrix^{-1} =
    \left ( {\bs{F}^T\ncov^{-1}\bs{F}} + \prcov^{-1} \right)^{-1}
    \approx \prcov^{1/2} (\bs{V}_r \bs{\Lambda}_r \bs{V}^T_r + \bs{I})^{-1}
    \prcov^{1/2} 
% - \bs{V}_r \bs{D}_r \bs{V}^T_r
    \]
  \end{block}
  \end{minipage}

  \vspace{0.15in}
  where $\bs{V}_r$ and $\bs{\Lambda}_r$ are the eigenvectors/values of
%  the eigenvalue problem 
$\bs{F}^T\ncov^{-1}\bs{F} \bs{v}_i = \lambda_i \prcov^{-1} \bs{v}_i $
  %, \bs{\Lambda}_r = \diag(\lambda_i)_{i=1}^r$
%  are truncated
%  generalized eigenvectors of the data misfit
%  Hessian, $\bs{D}_r =\diag(\lambda_i/(\lambda_i+1))$, and used
%  the Sherman-Morrison-Woodbury th. % to invert/factor.

\item Spectrum of the prior-preconditioned likelihood Hessian for the
  Arolla glacier (139 parameters, left) and
  Antarctica (1.19M parameters, right) 
%
\begin{figure}[htb]
    \centering\includegraphics[width=0.35\textwidth]{extraplots/spectrum.pdf}
    \hspace{0.2in}
    \centering\includegraphics[width=0.35\textwidth]{extraplots/spec_ppmisfit_hess_coarseandfine_new.pdf}
  \end{figure}
  %
\end{itemize}
\end{frame}
%---------------------------------------------------------------------------------%
\begin{frame}
 \frametitle{Exploiting the problem structure}
 \framesubtitle{Taking advantage of the parameter-to-observable mapping}
 \center

   \only<1>{
 \begin{tikzpicture}
 \fill[red!90,nearly transparent] (5,0) -- (7,.5) -- (3,.5) -- (1,0) -- cycle;
\draw [fill=red!40!white,opacity=1] (3,2.25) -- (3.5,2.25) -- (3.25,.25) -- cycle;
\draw [fill=red] (3.25,2.25) circle (.25cm and 0.07cm);
\draw (3.25,2.25) -- (2,3);
\draw [thick] (2,3) node[above]{sensitivities, $\frac{\partial u}{\partial\beta_j}$};
\draw[red,dashed] (7,.5) -- (3,.5) -- (1,0);
\draw [fill=blue!40!white,opacity=1] (4.25,.25) -- (4.5,2.25) -- (4.75,.25) -- cycle;
\draw [fill=blue] (4.5,.25) circle (.25cm and 0.07cm);
\draw (4.5,.25) -- (2,-.5);
\draw [thick] (2,-.5) node[below]{sensitivities, $\frac{\partial u_i}{\partial\beta}$};
\draw (1,0) -- (5,0) -- (5,2) -- (1,2) -- (1,0);
\draw  (5,2) -- (7,2.5) -- (3,2.5) -- (1,2);
\draw  (7,2.5) -- (7,.5) -- (5,0);
\draw  (5,2.25) -- (6,3) ;
\draw [thick] (6,3) node[above]{measurements, $\boldsymbol{d}$};
\draw[red]   (5.5,.25) -- (5.5,.125) ;
\draw  (5.5,.125) -- (5.5,-.5) ;
\draw [thick] (5.5,-.5) node[below]{parameter, $\beta(x)$};

\draw[dashed] (3,.5) -- (3,2);
\draw[blue,dashed] (3,2) -- (3,2.5);
\fill[blue!90,nearly transparent] (5,2) -- (7,2.5) -- (3,2.5) -- (1,2) -- cycle;
\end{tikzpicture}
   }
      \only<2>{
     \begin{center}
       \includegraphics[width=0.6\columnwidth]{extraplots/Hess1_ordered.png}
     \end{center}
   }
\begin{itemize}
\item We expect (hope) for narrow sensitivities
\item $\Rightarrow$ We hope that
  $\boldsymbol{\hat{H}}:=\boldsymbol{\prcov^{-1/2}} { \bs
  F }^T\boldsymbol{\ncov^{-1}} { \bs F
}\boldsymbol{\prcov^{-1/2}}+ { \bs I }$ is concentrated on
  the diagonal
\end{itemize}
\end{frame}
%---------------------------------------------------------------------------------%
\begin{frame}
  \frametitle{$\mathcal{H}$-matrices}

  . . . add description of H-matrices . . .

  . . . discuss computational cost . . .

  . . . add reference and link to the H-matrix library that Nick used . . .
  
  %\framesubtitle{Hierarchical off-diagonal low-rank (HODLR) matrices}
%  \small
%  \vspace{-0.1in}
%  \begin{align}
%    \bs{H}_{\text{\tiny HODLR}}=&
%    \begin{bmatrix}
%      \boldsymbol{A}_{1}^{(1)} & \textcolor{red}{\boldsymbol{B}_{1}^{(1)}}\\
%      \textcolor{red}{\boldsymbol{B}_{2}^{(1)}} & \boldsymbol{A}_{2}^{(1)}
%    \end{bmatrix}\nonumber\hspace{5.5cm}\textcolor{red}{\bullet}\textrm{ Low-rank }\\
%    =&\left[
%      \begin{array}{c;{2pt/2pt}c}
%        \begin{array}{cc}\boldsymbol{A}_{1}^{(2)} &  \textcolor{red}{\boldsymbol{B}_{1}^{(2)}}\\
%          \textcolor{red}{\boldsymbol{B}_{2}^{(2)}} &  \boldsymbol{A}_{2}^{(2)}
%        \end{array} &  \textcolor{red}{\boldsymbol{B}_{1}^{(1)}}\\ \hdashline[2pt/2pt]
%        \textcolor{red}{\boldsymbol{B}_{2}^{(1)}} &
%        \begin{array}{cc}\boldsymbol{A}_{3}^{(2)} &   \textcolor{red}{\boldsymbol{B}_{3}^{(2)}}\\
%          \textcolor{red}{\boldsymbol{B}_{4}^{(2)}} &  \boldsymbol{A}_{4}^{(2)}
%      \end{array} \end{array}\right]\nonumber\\
%    =&\left[
%      \begin{array}{c;{2pt/2pt}c}
%        \begin{array}{c;{2pt/2pt}c}
%          \begin{array}{cc}\boldsymbol{A}_{1}^{(3)} &  \textcolor{red}{\boldsymbol{B}_{1}^{(3)}}\\  \textcolor{red}{\boldsymbol{B}_{2}^{(3)}}  & \boldsymbol{A}_{2}^{(3)}
%          \end{array} &  \textcolor{red}{\boldsymbol{B}_{1}^{(2)}}\\ \hdashline[2pt/2pt]
%          \textcolor{red}{\boldsymbol{B}_{2}^{(2)} }&
%          \begin{array}{cc}\boldsymbol{A}_{3}^{(3)} &  \textcolor{red}{\boldsymbol{B}_{3}^{(3)}}\\
%            \textcolor{red}{\boldsymbol{B}_{4}^{(3)}}  & \boldsymbol{A}_{4}^{(3)}
%          \end{array}
%        \end{array} &  \textcolor{red}{\boldsymbol{B}_{1}^{(1)}}\\  \hdashline[2pt/2pt]
%        \textcolor{red}{\boldsymbol{B}_{2}^{(1)}} &
%        \begin{array}{c;{2pt/2pt}c}
%          \begin{array}{cc}\boldsymbol{A}_{5}^{(3)} &  \textcolor{red}{\boldsymbol{B}_{5}^{(3)}}\\
%            \textcolor{red}{\boldsymbol{B}_{6}^{(3)}}  & \boldsymbol{A}_{6}^{(3)}
%          \end{array} &  \textcolor{red}{\boldsymbol{B}_{3}^{(2)}}\\  \hdashline[2pt/2pt]
%          \textcolor{red}{\boldsymbol{B}_{4}^{(2)}} &
%          \begin{array}{cc}\boldsymbol{A}_{7}^{(3)} &  \textcolor{red}{\boldsymbol{B}_{7}^{(3)}}\\  \textcolor{red}{\boldsymbol{B}_{8}^{(3)}}  & \boldsymbol{A}_{8}^{(3)}
%          \end{array}
%        \end{array}
%      \end{array}
%      \right]\nonumber
%  \end{align}
%  \vspace{-1mm}
%  \begin{itemize}
%  \item [] \scriptsize{Details in: J. Ballani, D. Kressner. {\em
%      Matrices with Hierarchical Low-Rank Structures}, Exploiting
%    Hidden Structure in Matrix Computations: Algorithms and
%    Applications, 2016}
%  \end{itemize}
\end{frame}
%--------------------------------------------------------------------------------
%\begin{frame}
%  \frametitle{Hierarchical off-diagonal low-rank (HODLR) matrices}
%  \framesubtitle{Graphical representation for refinement levels $L=1,2,3$}
%
%  \begin{columns}
%    \begin{column}{0.33\paperwidth}
%		\begin{figure}[width=0.75\columnwidth]{
%	\begin{tikzpicture}
%	\begin{scope}[yscale=0.2, xscale=0.2]
%	\pgfmathtruncatemacro{\L}{1};
%	\pgfmathtruncatemacro{\scalefac}{16/(2^\L)};
%	\pgfmathtruncatemacro{\shiftval}{2^(\L-1)};
%	\begin{scope}[yscale=-1, xscale=1]
%	\begin{scope}[yshift=-\shiftval, xshift=-\shiftval]
%	\begin{scope}[yscale=\scalefac, xscale=\scalefac]
%	\pgfmathtruncatemacro{\msize}{2^\L};
%	%\pgfmathtruncatemacro{\gridstep}{2^(-\L)}
%	\pgfmathtruncatemacro{\colstep}{80};
%	%\pgfmathtruncatemacro{\colstep}{80/(\L-1)};
%	%\draw[step=\gridstep cm, gray, very thin] (0, 0) grid (\msize, \msize);
%	\foreach \l in {1,...,\L}
%		{
%		\pgfmathtruncatemacro{\delI}{\msize*2^(-\l)};
%		\pgfmathtruncatemacro{\DelI}{2*\delI};
%		%\pgfmathtruncatemacro{\colorl}{10+(\l-1)*\colstep};
%		\pgfmathtruncatemacro{\colorl}{75};
%		\pgfmathtruncatemacro{\maxi}{2^(\l-1)};
%		\foreach \i in {1,...,\maxi}
%			{
%			\pgfmathtruncatemacro{\a}{(\i-1)*\DelI}
%			\pgfmathtruncatemacro{\b}{(\i-1)*\DelI+\delI}
%			\pgfmathtruncatemacro{\c}{\i*\DelI}
%			\filldraw[fill=red!\colorl!white] (\a, \b) rectangle (\b, \c);
%			\filldraw[fill=red!\colorl!white] (\b, \a) rectangle (\c, \b);
%			}
%		}
%		% diagonal blocks
%	\pgfmathtruncatemacro{\delI}{\msize*2^(-\L)};
%	\pgfmathtruncatemacro{\DelI}{2*\delI};
%	\pgfmathtruncatemacro{\colorl}{20+(\L-1)*\colstep};
%	\pgfmathtruncatemacro{\maxi}{2^(\L-1)};
%	\foreach \i in {1,...,\maxi}
%		{
%		\pgfmathtruncatemacro{\a}{(\i-1)*\DelI}
%		\pgfmathtruncatemacro{\b}{(\i-1)*\DelI+\delI}
%		\pgfmathtruncatemacro{\c}{\i*\DelI}
%		\filldraw[fill=black] (\a, \a) rectangle (\b, \b);
%		\filldraw[fill=black] (\b, \b) rectangle (\c, \c);
%		}
%	\end{scope}
%	\end{scope}
%	\end{scope}
%	\end{scope}
%\end{tikzpicture}
%}
%		\end{figure}
%	%\end{center}
%%\end{minipage}
%    \end{column}
%    \begin{column}{0.33\paperwidth}
%%\begin{minipage}{.49\columnwidth}
%	%\begin{center}
%		\begin{figure}[width=0.75\columnwidth]{
%	\begin{tikzpicture}
%	\begin{scope}[yscale=0.2, xscale=0.2]
%	\pgfmathtruncatemacro{\L}{2};
%	\pgfmathtruncatemacro{\scalefac}{16/(2^\L)};
%	\pgfmathtruncatemacro{\shiftval}{2^(\L-1)};
%	\begin{scope}[yscale=-1, xscale=1]
%	\begin{scope}[yshift=-\shiftval, xshift=-\shiftval]
%	\begin{scope}[yscale=\scalefac, xscale=\scalefac]
%	\pgfmathtruncatemacro{\msize}{2^\L};
%	%\pgfmathtruncatemacro{\gridstep}{2^(-\L)}
%	\pgfmathtruncatemacro{\colstep}{40/(\L-1)};
%	%\draw[step=\gridstep cm, gray, very thin] (0, 0) grid (\msize, \msize);
%	\foreach \l in {1,...,\L}
%		{
%		\pgfmathtruncatemacro{\delI}{\msize*2^(-\l)};
%		\pgfmathtruncatemacro{\DelI}{2*\delI};
%		%\pgfmathtruncatemacro{\colorl}{50+(\l-1)*\colstep};
%		\pgfmathtruncatemacro{\colorl}{75};
%		\pgfmathtruncatemacro{\maxi}{2^(\l-1)};
%		\foreach \i in {1,...,\maxi}
%			{
%			\pgfmathtruncatemacro{\a}{(\i-1)*\DelI}
%			\pgfmathtruncatemacro{\b}{(\i-1)*\DelI+\delI}
%			\pgfmathtruncatemacro{\c}{\i*\DelI}
%			\filldraw[fill=red!\colorl!white] (\a, \b) rectangle (\b, \c);
%			\filldraw[fill=red!\colorl!white] (\b, \a) rectangle (\c, \b);
%			}
%		}
%		% diagonal blocks
%	\pgfmathtruncatemacro{\delI}{\msize*2^(-\L)};
%	\pgfmathtruncatemacro{\DelI}{2*\delI};
%	\pgfmathtruncatemacro{\colorl}{20+(\L-1)*\colstep};
%	\pgfmathtruncatemacro{\maxi}{2^(\L-1)};
%	\foreach \i in {1,...,\maxi}
%		{
%		\pgfmathtruncatemacro{\a}{(\i-1)*\DelI}
%		\pgfmathtruncatemacro{\b}{(\i-1)*\DelI+\delI}
%		\pgfmathtruncatemacro{\c}{\i*\DelI}
%		\filldraw[fill=black] (\a, \a) rectangle (\b, \b);
%		\filldraw[fill=black] (\b, \b) rectangle (\c, \c);
%		}
%	\end{scope}
%	\end{scope}
%	\end{scope}
%	\end{scope}
%\end{tikzpicture}
%}
%		\end{figure}
%    \end{column}
%	  \begin{column}{.33\paperwidth}
%		\begin{figure}[width=0.75\columnwidth]{
%	\begin{tikzpicture}
%	\begin{scope}[yscale=0.2, xscale=0.2]
%	\pgfmathtruncatemacro{\L}{3};
%	\pgfmathtruncatemacro{\scalefac}{16/(2^\L)};
%	\pgfmathtruncatemacro{\shiftval}{2^(\L-1)};
%	\begin{scope}[yscale=-1, xscale=1]
%	\begin{scope}[yshift=-\shiftval, xshift=-\shiftval]
%	\begin{scope}[yscale=\scalefac, xscale=\scalefac]
%	\pgfmathtruncatemacro{\msize}{2^\L};
%	%\pgfmathtruncatemacro{\gridstep}{2^(-\L)}
%	\pgfmathtruncatemacro{\colstep}{40/(\L-1)};
%	%\draw[step=\gridstep cm, gray, very thin] (0, 0) grid (\msize, \msize);
%	\foreach \l in {1,...,\L}
%		{
%		\pgfmathtruncatemacro{\delI}{\msize*2^(-\l)};
%		\pgfmathtruncatemacro{\DelI}{2*\delI};
%		%\pgfmathtruncatemacro{\colorl}{50+(\l-1)*\colstep};
%		\pgfmathtruncatemacro{\colorl}{75};
%		\pgfmathtruncatemacro{\maxi}{2^(\l-1)};
%		\foreach \i in {1,...,\maxi}
%			{
%			\pgfmathtruncatemacro{\a}{(\i-1)*\DelI}
%			\pgfmathtruncatemacro{\b}{(\i-1)*\DelI+\delI}
%			\pgfmathtruncatemacro{\c}{\i*\DelI}
%			\filldraw[fill=red!\colorl!white] (\a, \b) rectangle (\b, \c);
%			\filldraw[fill=red!\colorl!white] (\b, \a) rectangle (\c, \b);
%			}
%		}
%		% diagonal blocks
%	\pgfmathtruncatemacro{\delI}{\msize*2^(-\L)};
%	\pgfmathtruncatemacro{\DelI}{2*\delI};
%	\pgfmathtruncatemacro{\colorl}{20+(\L-1)*\colstep};
%	\pgfmathtruncatemacro{\maxi}{2^(\L-1)};
%	\foreach \i in {1,...,\maxi}
%		{
%		\pgfmathtruncatemacro{\a}{(\i-1)*\DelI}
%		\pgfmathtruncatemacro{\b}{(\i-1)*\DelI+\delI}
%		\pgfmathtruncatemacro{\c}{\i*\DelI}
%		\filldraw[fill=black] (\a, \a) rectangle (\b, \b);
%		\filldraw[fill=black] (\b, \b) rectangle (\c, \c);
%		}
%	\end{scope}
%	\end{scope}
%	\end{scope}
%	\end{scope}
%\end{tikzpicture}
%}
%		\end{figure}
%	  \end{column}
%	  \end{columns}
%%\begin{minipage}{.49\columnwidth}
%        %\end{minipage}
%
%        \vspace{1in}
%        \begin{itemize}
%        \item [] \scriptsize{Details in: J. Ballani, D. Kressner. {\em
%            Matrices with Hierarchical Low-Rank Structures}, Exploiting
%          Hidden Structure in Matrix Computations: Algorithms and
%          Applications, 2016}
%        \end{itemize}
%\end{frame}
%%------------------------------------------------------------------------------
%-------------------------------------------------------------%
\begin{frame}
  \frametitle{$\mathcal{H}$-matrices}
  \framesubtitle{Computational cost}

  . . .  Tucker, please adapt this slide if you can . . .
  
\noindent \textbf{HODLR Compression/Approximation:}
\begin{itemize}
\item Cost: $\mathcal{O} (k \log N)$ Hessian-applies, where $k$ is the
  off-diagonal (local) rank, and $N$ is the parameter dimension.
\item Apply $\bs{H}$ to random vectors with specified null entries to
  sample the column spaces of the off-diagonal blocks.
\vspace{1mm}

The samples are likely aligned with dominant modes, thus affording low
rank approximations.

\item [] \scriptsize{Details in: P.G. Martinsson, {\em Compressing
    Rank-Structured Matrices Via Randomized Sampling}, SIAM Journal on
  Scientific Computing, 2016}
\end{itemize}
\vspace{1mm}

\noindent \textbf{Inverse Hessian-apply / Posterior covariance apply:}
\begin{itemize}
\item At $\mathcal{O}\left(k^2 N\,\log^{2} N \right)$ cost, one can
  factor the $L$ times refined $\boldsymbol{H}$ as:
  $\boldsymbol{H}_{\rm
    HODLR}=\boldsymbol{K}_{L}\boldsymbol{K}_{L-1}\boldsymbol{K}_{L-2}\cdots\boldsymbol{K}_{1}\boldsymbol{K}_{0}$,
  where

$\bs{K}_{i}$, $i\in\left\{0,1,\dots,L-1\right\}$ is block diagonal and
  each diagonal block is a low-rank update of the identity.
\item Making gratuitous use of the Sherman-Morrison-Woodbury formula
  one has a $\mathcal{O}\left(k \, N\,\log N \right)$ means of
  applying $\left(\bs{H}_{\rm HODLR}\right)^{-1}$ to vectors.
\item [] \scriptsize{Details in: S. Ambikasaran, E. Darve, {\em An
    $\mathcal{O}(N\, \log N)$ Fast Direct Solver for Partial
    Hierarchically Semi-Separable Matrices}, Springer Journal of
  Scientific Computing, 2013}
\end{itemize}
\vspace{1mm}
\end{frame}
%--------------------------------------------------------------------------
\begin{frame}
  \frametitle{Summary}

  \begin{itemize}
  \item Hessian operators arising in inverse problems governed by
    partial differential equations play an essential role in
    delivering efficient, dimension-independent methods for both
    \begin{itemize}
    \item Newton solution of deterministic inverse problems (i.e., for
      computing the MAP point);
      \vspace{0.05in}
    \item Markov chain Monte Carlo sampling to characterize the full
      posterior.
    \end{itemize}
    \vspace{0.05in}
  \item Low-rank approximations of the Hessian are effective when the
    data contain limited information about the parameters but become
    prohibitive as the data becomes more informative (as is the case
    for ice sheet inverse problems).
    \vspace{0.05in}
  \item Hierarchical matrix representations promise a more efficient
    Hessian approximation, in the case of a low noise (i.e., higher
    effective rank, large informative data) regime.
  \end{itemize}

  \begin{itemize}
  \item [] \scriptsize{Details in: T. Hartland, G. Stadler, and
    N. Petra. {\em Hierarchical off-diagonal low-rank (HODLR)
      approximation for Hessians in Bayesian inference with
      application to ice sheet models}, In preparation.}
  \end{itemize}
\end{frame}
\end{document}
