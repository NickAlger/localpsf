\documentclass[11pt]{article}
\renewcommand*\familydefault{\sfdefault}
%\usepackage{amssymb,amsmath,amsfonts,comment}
%\usepackage{amsmath,amssymb,graphicx,subfigure,psfrag}
\usepackage{amsmath,amssymb,graphicx,subfigure,psfrag,upgreek}
\usepackage{amssymb,mathrsfs}
\usepackage[margin=1in]{geometry}
\usepackage{tikz,pgfplots,graphicx}
\usepackage{color,pdfcolmk}
\usepackage{amsmath}

%\newcommand{\alennote}[1]{\noindent\emph{\textcolor{blue!50!black}{Alen:#1\:}}}
%\newcommand{\gsnote}[1]{\noindent\emph{\textcolor{green}{GS: #1\:}}}
\newcommand{\todo}[1]{\noindent\emph{\textcolor{red}{Todo: #1\:}}}
\newcommand{\note}[1]{\noindent\emph{\vspace{1ex}\textcolor{blue}{Note: #1\:}}\\[1ex]}
\newcommand{\nnote}[1]{\noindent\emph{\vspace{1ex}\textcolor{red}{Note: #1\:}}}
\newcommand{\referee}[1]{\vspace{.1ex}\noindent{\textcolor{blue}{#1}}}
\newcommand{\nnn}{\mathbf{n}}
\newcommand{\fff}{\mathbf{f}}
\newcommand{\uuu}{\mathbf{u}}
\newcommand{\+}{\otimes}
\newcommand{\m}{\mathcal}
\newcommand{\bs}[1]{\ensuremath{\boldsymbol{#1}}}
\newcommand{\eps}{\varepsilon}


\newcommand{\gs}[1]{\textcolor{green}{G: #1}}
\newcommand{\np}[1]{\textcolor{blue}{N: #1}}
\newcommand{\tuck}[1]{\textcolor{brown}{T: #1}}
\newcommand{\mauro}[1]{\textcolor{red}{#1}}



\begin{document}

\section*{Reply to the Reviewer 2:}

Thanks for the careful reading and for your helpful comments and
suggestions.  Please find below point-by-point replies (in black) to
your comments and questions (which are reprinted in blue). To give you
an overview of all the changes in the paper, we also provide a
diff-document that highlights the changes between the initial
submission and this re-submission.\\[1ex]

\referee{This article addresses an important topic in large-to-extreme-scale inverse problems. Namely, efficient approximation of the Hessian of the negative-log of the posterior in Bayesian inversion.
Specifically, this work explores utilizing hierarchical off-diagonal low-rank (HODLR) matrix approximations of the Hessian.
The idea is clearly presented and the manuscript is well-written, however, I believe some---I would say minor---adjustments are required before recommending acceptance for publications.
Please see my comments below. \\}

\paragraph{\sc General Comments:}

\begin{enumerate} 
\item \referee{I do not think the terms ``Gaussianization, Gaussianized, ...'' are properly utilized in this work. The term Gaussianization has been actively used, especially in the machine learning literature, to refer to transformation of non-Gaussian data/variable in to a Gaussian. This requires finding a map/transformation between the original distribution and a Gaussian, e.g., an optimal transport, or normalizing flows. The authors, however, use these terms to denote approximating a non-Gaussian distribution with a localized Gaussian, e.g., following a Laplacian approach.}

\noindent{We appreciate drawing our attention to the fact that our usage of ``Gaussianization, Gaussianized...'', would be unclear to many readers. We have updated our usage to ``the Laplace approximation of the posterior''.}

\item \referee{Re: Algorithm 2 (page 22); it would be great to discuss how $r_1, \ldots, r_{L}$ can be auto tuned. What metric can be used to automatically choose the optimal value of $L$ and then choose the rank for each level automatically.}

\noindent{Done, with regard to choosing the ranks $r_{1},\dots,r_{L}$. We are not aware of an automatic way to choose the optimal value of $L$ but have described the components that we believe could be used to develop such means.}
\end{enumerate} 

\paragraph{\sc Other comments:}
\begin{enumerate}
\item  \referee{Page 1, Line 25: ``approximation of Hessian applies...''; Hessian of what?}

\noindent{We now use the more descriptive ``discretized objective functional Hessian vector-products''.}

\item \referee{Page 1, Line 28--31: statement too long, maybe try splitting!}

\noindent{Done.}

\item \referee{Page 1, Line 29: ``sampling from Gaussian proposal distributions'' $\rightarrow$ ``sampling Gaussian distributions''.}

\noindent{We have not used the exact suggested text since the sentence phrasing has been changed. We believe that ``...that provide a means to sample Gaussian distributions...'', is as close to the suggested rephrasing but which maintains consistency with the new sentence structure.}

\item \referee{Page 1, Line 30: I am not sure I understand what the authors mean by ``Newton solves by direct linear methods''!}

\noindent{We had intended to describe methods such as LU factorization, which is known as a direct method of linear system solution. We hope that by referencing a relevant factorization that the message intended to be communicated is now made more clear.}

\item \referee{Page 1, Line 46: ``to inverse problems'' ``with inverse problems''.}

\noindent{Done, thanks!}

\item \referee{Page 2, Line 7: simulation... systems plays'' $\rightarrow$ ``simulations ... systems play''.}

\noindent{Done.}

\item \referee{Page 3, Line 17: ``generate approximate samples from a Gaussian posterior distribution for...'' $\rightarrow$ ``generate samples from an approximate Gaussian posterior for...''}

\noindent{Done, but we have kept ``Gaussian posterior distribution for...''}

\item \referee{Page 3, Line 22: ``including for a Greenland ice...'' $\rightarrow$ ``including a Greenland ice...''}

\noindent{Done.}

\item \referee{Page 3, Line 28: Add a reference; ``In this section,'' $\rightarrow$ ``In this section (2.1),''}

\noindent{Done, but with the capitalization scheme used throughout the manuscript, ``In this Section (2.1),''}

\item \referee{Page 3, line 44: Remove ``i.e., the data,'' as it is redundant.}

\noindent{Done. We have added an additional sentence in which we explicitly state that we will later refer to $\bs{d}$ as the data instead of the more verbose observational data.}

\item \referee{Page 3, Line 46--47: ``and mathematically'' $\rightarrow$ ``and is mathematically''}

\noindent{Done.}

\item \referee{Page 4, Lines 12--13: The authors use the term ``parameter-to-PDE-solution map'' which has not been introduced. Only the combined ``parameter-to-observable map'' has been introduced on line $10$ above. I suggest being consistent and either explicitly decompose the parameter-to-observable-map into a solution operator (solution-to-PDE-solution) and an observation operator, or just restrict usage to ``parameter-to-observable-map''. Either one or both of the solution operator and the observation operator can be nonlinear and each is associated with its own difficulties.}

\noindent{Done. We have restricted usage to the ``parameter-to-observable map''.}

\item \referee{Page 4, Line 27: ``it's gradient'' $\rightarrow$ ``its gradient''}

\noindent{Done.}

\item \referee{Page 4, Lines 28--33: This paragraph requires some
  modification. If you can well-approximate the posterior you do not
  necessarily need MCMC; you can even use naive MC by applying MH step
  to samples from this Gaussian. MCMC does not generally require
  approximating the posterior with a Gaussian, however sampling a
  Gaussian approximation can be useful. For example in Hamiltonian
  Monte Carlo (HMC), samples from a Gaussian approximation can be used
  to build a mass matrix that optimizes the performance of the
  sampler; diagonal of the mass matrix are optimally set to posterior
  precisions (or approximations thereof). See e.g., reference
  \cite{neal2011mcmc, attia2017reduced}.}

\noindent{We agree with the reviewer, but our experience is that using
  MCMC is still superior to using MH (or importance sampling)
  directly, in particular in high dimensions.  We adjusted this
  paragraph in the revised manuscript.}
  
%\todo{Georg, please see if you you'd like to add or change anything here.}

\item \referee{Page 4, Line 38: define $_{\ell_{2}}$ presented in the previous equation}

\noindent{This item has been addressed by removing any usage of $_{\ell_{2}}$.}

\item \referee{Page 5, Line 38: Maybe redefine paragraph titles to add a period at the end; ``Compression'' $\rightarrow$ ``Compression.''}

\noindent{Done.}

\item \referee{Page 5, Line 50: replace the comma with a semicolon.}

\noindent{Done.}

\item \referee{Page 5, Line 60: ``algorithm is made possible''; where? add a reference/citation here!}

\noindent{Done.}

\item \referee{Page 6, Line 5: a period is missing after ``Approximations''}

\noindent{Done.}

\item \referee{Page 6, Line 6: hereafter use a different symbol for the number of oversampling vectors as $\bs{d}$ has been utilized to denote the data, despite being boldface.}

\noindent{Done. We are now using the symbol $q$ to denote the number of oversampling vectors.}

\item \referee{Page 6, Lines 24--30: this sentence is too long; try splitting.}

\noindent{We have reformulated this sentence.}

\item \referee{Page 7, Line 40: the term data-misfit Hessian has already been introduced before; remove ``i.e., the Hessian ... functional'' as it is redundant.}

\noindent{Done.}

\item \referee{Page 7, Line 44: remove ``of it''}

\noindent{Done.}

\item \referee{Page 8, Line 40: ``basis function'' $\rightarrow$ ``basis functions''}

\noindent{Done.}

\item \referee{Page 8, Line 51: please see my comment above about using the term ``Gaussianized''}

\noindent{Noted and addressed.}

\item \referee{Page 9, Line 22: I am not sure what ``means of applying'' here means!}

\noindent{We have clarified the vague ``means of applying'' with ``means of computing square root and inverse square root matrix-vector products.}

\item \referee{Page 9, Lines 33--40: this paragraph is redundant. This material/argument has already been introduced in Section 2.1.}

\noindent{We believe that this material is important as it helps show how the example problem fits into the larger class of inverse problems, therefore we decided to keep it here.}

%\todo{I (Tucker) believe that this argument could be strengthened.}

\item \referee{Page 9, Line 49: ``can mathematically'' $\rightarrow$ ``can be mathematically''}

\noindent{Done.}

\item \referee{Page 10 I suggest using subequations for the model Equations (6)--(9)}

Done.

\item \referee{Page 10, Lines 16--19, 22, 43: in reference to model equations, use $\backslash$eqref instead of $\backslash$ref}
	
Done.

\item \referee{Page 11, Line 5: see my comment above about using the term ``Gaussianizaiton''. I will not comment on this hereafter.}

\noindent{Noted and addressed.}	

\item \referee{Page 11, Lines 21, 28: use $\backslash$eqref in reference to model equations.}

Done.

\item \referee{Page 11, Line 31: ``add $1\%$ relative Gaussian noise...'': $1\%$ of what? mean true observation magnitude, $\ell_{\infty}$, ...etc.; please clarify!}

\noindent{Done.}

\item \referee{Page 11, Line 57: I do not believe ``approximate samples'' is a correct terminology. I suggest using ``samples from the approximate posterior''.}

\noindent{This has been addressed.}

\item \referee{Page 12, Figure 3 caption: something is not quite right here. The caption states that circles show representative (random measurement locations), however, lines 29-30 on page 11 say that observations are uniformly distributed. What am I missing here? }

\noindent{The figure has been adjusted so that the measurement locations are now illustrated as being uniformly distributed.}

\item \referee{Page 14, Line 57: there is an extra comma after ``more data''}

\noindent{Done.}

\item \referee{Page 15, Figure 7 caption (line 21): ``low-rank (LR)'' $\rightarrow$ ``LR'' as the acronym has already been introduced; same on line 24 of the same caption: ``low-rank'' $\rightarrow$ ``LR'' }

\noindent{Done.}

\item \referee{Page 15, Figure 8 caption: same as the comment above on using ``LR''}

\noindent{Done.}

\item \referee{Page 16, Line 5: remove the comma after ``Albany'', and remove ``so-called''}

\noindent{Done.}

\item \referee{Page 16, Line 30: use $\backslash$eqref in reference to Equation (3)}

Done.

\item \referee{Page 20, Line 29: remove the comma before ``associated to finite...''}

\noindent{Done.}

\item \referee{Page 23, Line  5: ``low-rank'' $\rightarrow$ ``LR'' }

\noindent{Done.}

\end{enumerate}
%\bibliographystyle{unsrt}
\bibliographystyle{iopart-num}
\bibliography{reviewer2_references}

%\noindent [1]- Neal, Radford M. "MCMC using Hamiltonian dynamics." Handbook of markov chain monte carlo 2, no. 11 (2011): 2.

%\noindent [2]- Attia, Ahmed, Razvan Stefanescu, and Adrian Sandu. "The reduced-order hybrid Monte Carlo sampling smoother." International Journal for Numerical Methods in Fluids 83, no. 1 (2017): 28-51.
\end{document}
