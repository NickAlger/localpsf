\documentclass[11pt]{article}
\renewcommand*\familydefault{\sfdefault}
%\usepackage{amssymb,amsmath,amsfonts,comment}
%\usepackage{amsmath,amssymb,graphicx,subfigure,psfrag}
\usepackage{amsmath,amssymb,graphicx,subfigure,psfrag,upgreek}
\usepackage{amssymb,mathrsfs}
\usepackage[margin=1in]{geometry}
\usepackage{tikz,pgfplots,graphicx}
\usepackage{color,pdfcolmk}
\usepackage{amsmath}
\usepackage{hyperref}

%\newcommand{\alennote}[1]{\noindent\emph{\textcolor{blue!50!black}{Alen:#1\:}}}
%\newcommand{\gsnote}[1]{\noindent\emph{\textcolor{green}{GS: #1\:}}}
\newcommand{\todo}[1]{\noindent\emph{\textcolor{red}{Todo: #1\:}}}
\newcommand{\note}[1]{\noindent\emph{\vspace{1ex}\textcolor{blue}{Note: #1\:}}\\[1ex]}
\newcommand{\nnote}[1]{\noindent\emph{\vspace{1ex}\textcolor{red}{Note: #1\:}}}
\newcommand{\referee}[1]{\vspace{.1ex}\noindent{\textcolor{blue}{#1}}}
\newcommand{\nnn}{\mathbf{n}}
\newcommand{\fff}{\mathbf{f}}
\newcommand{\uuu}{\mathbf{u}}
\newcommand{\+}{\otimes}
\newcommand{\m}{\mathcal}
\newcommand{\bs}[1]{\ensuremath{\boldsymbol{#1}}}
\newcommand{\eps}{\varepsilon}


\newcommand{\gs}[1]{\textcolor{green}{G: #1}}
\newcommand{\np}[1]{\textcolor{blue}{N: #1}}
\newcommand{\tuck}[1]{\textcolor{brown}{T: #1}}
\newcommand{\mauro}[1]{\textcolor{red}{#1}}



\begin{document}

\section*{Reply to the Reviewer 2:}

Thanks for the careful reading and for your helpful comments and
suggestions.  Please find below point-by-point replies (in black) to
your comments and questions (which are reprinted in blue). To give you
an overview of all the changes in the paper, we also provide a
diff-document that highlights the changes between the initial
submission and this re-submission.\\[1ex]

\referee{I have read the paper with pleasure and have the following
  suggestions for improvement:\\}

\begin{enumerate}
\item \referee{Interesting work, generally well-written. Perhaps a
  small illustrative example with a few figures at the beginning could
  help understand the procedure intuitively.}

  \noindent{To do.}

\item \referee{possibly interesting references for estimating Hessians
  in linearised seismic imaging include:
  \url{https://doi.org/10.1016/j.acha.2007.06.007},
  \url{https://doi.org/10.1190/1.2836323}}

  \noindent{We added text in the paragraph at the end of Section 2 to discuss the (lack of) applicability of the method to wave inverse problems and added these references.}

\item \referee{How about comparison with other methods, like those
  proposed in [7],[52]? and you say a bit more? it is hard to conclude
  that your method will always outperform these methods.}

    \noindent{For ref 7: Ask Tucker if we could do HODLR on the
      Hessian coming from ice mountain problem and record the number
      of mat-vecs needed to achieve a certain accuracy, something like
      Figure 8 in teh HODLR paper. Nick will do some estimates to
      compare things with ref 52.}

\item \referee{Perhaps it is illustrative to include some 1D blurring
  examples to illustrate robustness of method to departure from
  assumptions (positivity, locality) e.g. on 'indefinite' kernels or
  kernels with multiple modes.}

  \noindent{Use a 2D-deblurring problem and cook up a Hessian with
    entries given by be some modified Gaussian and show that the error
    increases as more negative entires are present in the kernel.}

 \item \referee{is code for applying the developed methods on other
   inverse problems available?}

  \noindent{Ideally we would like to release the code, however this
    will take time. In the meantime, we will make the code available
    upon request.}
  
\end{enumerate}


%\bibliographystyle{unsrt}
%\bibliographystyle{iopart-num}
%\bibliography{reviewer2_references}

\end{document}
