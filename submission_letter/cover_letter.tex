\documentclass[11pt]{article}
\setlength{\hoffset}{-1in}       \setlength{\voffset}{-1in}
\setlength{\textwidth}{140mm}    \setlength{\textheight}{240mm}
\setlength{\topmargin}{18mm} \setlength{\oddsidemargin}{35mm}
\setlength{\evensidemargin}{25mm} \setlength{\headheight}{25mm}
\setlength{\headsep}{5mm} \setlength{\marginparwidth}{20mm}


\usepackage{graphicx}
\usepackage[right]{eurosym}

\begin{document}

\pagestyle{empty}
 \begin{figure}
 \begin{minipage}{.5\columnwidth}
 \end{minipage}\hfill
 \begin{minipage}[b]{.48\columnwidth}
\flushright
 \noindent{\sc\large Nick Alger}\vspace{-0.9cm}
 \end{minipage}
 \end{figure}
 \noindent\rule{\columnwidth}{.5pt}\vspace{-2ex}
 \begin{figure}[h]
 \begin{minipage}{.2\columnwidth}
 \end{minipage}\hfill
 \begin{minipage}{.78\columnwidth}
 \flushright\vspace{-2ex}
 {\noindent Oden Institute for Computational Engineering \& Sciences \\ 
 University of Texas at Austin, \\ 
 1 Univ.\ Station, C0200,
 TX 78712 USA \\
 email: {\tt nalger225@gmail.com}\\
 tel: +1 425 736 4111}\\
 \end{minipage}
 \end{figure}
% \vspace{1cm} \noindent \\
\phantom{adafdsf}
\vspace{3ex}

%\noindent
%Jan S. Hesthaven \\
%EPFL SB-DO\\
%PH A2 364 (B\^atiment PH)\\
%Station 3\\
%CH-1015 Lausanne \\
%Switzerland
%{
%\flushright{\today}\\
%}
% \vspace{0.3cm}
\noindent Dear Professor Hans De Sterck:\\[2ex] 

We wish to submit the manuscript {\em Point spread function approximation of high rank Hessians with locally supported non-negative integral kernels} by Nick Alger, Tucker Hartland, Noemi Petra, and Omar Ghattas to be considered for publication in SISC in the {\em
  Numerical Algorithms for Scientific Computing} section.
%

In this manuscript, we propose an efficient point spread function (PSF) method for approximating matrix-free operators with locally supported non-negative integral kernels. We improve upon existing PSF methods by using a novel moment method for forming a-priori ellipsoid estimates for the supports of impulse responses. By solving an ellipsoid packing problem, we compute a large number of impulse responses per operator application, making the method highly efficient in terms of operator applications. We also provide an improved method for interpolating impulse responses based on a notion we call ``local mean displacement invariance,'' which generalizes classical local translation invariance. We apply the method to approximate high rank Hessians PDE constrained inverse problems. Our numerical results demonstrate that the proposed method outperforms conventional approaches, and that the method can approximate high rank Hessians with a small number of operator applies. The manuscript has not been submitted elsewhere for
publication.\\[1ex]


\noindent Thank you for your consideration.\\[3ex]


\noindent Sincerely,\\[5ex]
\phantom{dd}\hspace{2cm} Nick Alger
\end{document}
