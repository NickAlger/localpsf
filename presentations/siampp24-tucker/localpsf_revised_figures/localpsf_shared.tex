% SIAM Shared Information Template
% This is information that is shared between the main document and any
% supplement. If no supplement is required, then this information can
% be included directly in the main document.


% Packages and macros go here
\usepackage{lipsum}
\usepackage{amsfonts}
\usepackage{graphicx}
\usepackage{epstopdf}
\usepackage{amsmath}
\usepackage{subcaption}
\usepackage{pgfplots}


%\usepackage{amsmath,amsthm}
%\usepackage[noend]{algorithmic}
%\usepackage{algorithmic}
\usepackage[algo2e, ruled, noend, linesnumbered]{algorithm2e}
%\usepackage[algo2e, ruled, vlined]{algorithm2e}
\usepackage{bm}
\ifpdf
  \DeclareGraphicsExtensions{.eps,.pdf,.png,.jpg}
\else
  \DeclareGraphicsExtensions{.eps}
\fi

\usepackage{placeins}
\usepackage{multirow}

\usepackage{booktabs}
\usepackage{tikz}
%\usepackage{tabularray}
\usepackage[export]{adjustbox}

% Add a serial/Oxford comma by default.
\newcommand{\creflastconjunction}{, and~}
\newcommand{\nor}[1]{\left\|#1\right\|}

%\newcommand{\vertcentered}[1]{\begin{tabular}{l} #1 \end{tabular}}

% Used for creating new theorem and remark environments
\newsiamremark{remark}{Remark}
\newsiamremark{hypothesis}{Hypothesis}
\crefname{hypothesis}{Hypothesis}{Hypotheses}
\newsiamthm{claim}{Claim}
\newsiamthm{prop}{Proposition}
\newsiamthm{defn}{Definition}
\newsiamthm{thm}{Theorem}
\newsiamthm{cor}{Corollary}
\newsiamthm{lem}{Lemma}


\newcommand{\norm}[1]{\|#1\|}


% Sets running headers as well as PDF title and authors
\headers{High-rank PSF Hessian approximation}{N. Alger, T. Hartland, N. Petra, and O. Ghattas}

% Title. If the supplement option is on, then "Supplementary Material"
% is automatically inserted before the title.
%\title{Matrix-free point spread function approximation of operators with locally supported non-negative integral kernels, with application to Hessians in PDE constrained inverse problems\thanks{Submitted to the editors DATE.
%\funding{This research was supported by the National Science Foundation under Grant No. DMS-1840265 and CAREER-1654311, DOD grant FA9550-21-1-0084, and DOE grants DE-SC0021239 and DE-SC0019303.}}}
\title{Point spread function approximation of high rank Hessians with locally supported non-negative integral kernels\thanks{Submitted to the editors DATE.
		\funding{This research was supported by the National Science Foundation under Grant No. DMS-1840265 and CAREER-1654311, DOD grant FA9550-21-1-0084, and DOE grants DE-SC0021239 and DE-SC0019303.}}}


% Authors: full names plus addresses.
\author{Nick Alger\thanks{Oden Institute for Computational Engineering and Sciences, The University of Texas at Austin, Austin, TX 
  (\email{nalger@oden.utexas.edu}).}
\and Tucker Hartland\thanks{Department of Applied Mathematics, University of California, Merced, Merced, CA. 
	(\email{thartland@ucmerced.edu}, \email{npetra@ucmerced.edu}).}
\and Noemi Petra\footnotemark[3]
%\footnotemark[1].}
%\thanks{Department of Applied Mathematics, University of California, Merced, Merced, CA. (\email{npetra@ucmerced.edu}).}
\and Omar Ghattas\thanks{Oden Institute for Computational Engineering and Sciences and Walker Department of Mechanical Engineering, The University of Texas at Austin, Austin, TX 
	(\email{omar@oden.utexas.edu}).}}

\newcommand{\Aop}{\mathcal{A}}
\newcommand{\AopPc}{\mathcal{A}_\text{pc}}
\newcommand{\AopPcMesh}{\mathcal{A}_\text{pc}^h}

\newcommand{\Aker}{\Phi}
\newcommand{\AkerPc}{\Phi_\text{pc}}
\newcommand{\AkerPcMesh}{\Phi_\text{pc}^h}

\newcommand{\AkerPcMat}{\mathbf{\Phi}_\text{pc}}

\newcommand{\Amat}{\mathbf{A}}
\newcommand{\AmatPc}{\mathbf{A}_\text{pc}}
\newcommand{\AmatPcSym}{\mathbf{A}_\text{pc}^\text{sym}}
\newcommand{\AmatPcSymPlus}{\mathbf{A}_\text{pc}^{\text{sym}+}}

\newcommand{\diraccomb}{\xi}
\newcommand{\combresponse}{\eta}
\newcommand{\weakadmconst}{C}
\newcommand{\horizinterpolant}{\theta}
\newcommand{\interpolatedvalues}{b}
\newcommand{\firstgreens}{G}
\newcommand{\secondgreens}{F}
\newcommand{\impulseresponse}{\phi}
\newcommand{\convkernel}{\varphi}
\newcommand{\febasis}{\psi}
\newcommand{\ratfct}{R}
\newcommand{\ratpole}{\omega}
\newcommand{\ratcoeff}{c}
\newcommand{\massmatrix}{\mathbf{M}}
\newcommand{\spatialvol}{V}
\newcommand{\spatialmean}{\mu}
\newcommand{\spatialcov}{\Sigma}
\newcommand{\genericdistribution}{\rho}
\newcommand{\pointbatch}{S}
\newcommand{\pointinteractionmatrix}{S}
\newcommand{\gdim}{d}
\newcommand{\fedim}{N}
\newcommand{\convrank}{r}
\newcommand{\hrank}{k_h}
\newcommand{\nbatch}{n_b}
\newcommand{\numnbr}{k_{n}}
\newcommand{\ratord}{l}
\newcommand{\nsamplepts}{m}
\newcommand{\ptsonebatch}{s}
\newcommand{\classicalrank}{r}
\newcommand{\secondgreenscoeff}{\beta}
\newcommand{\constcoeff}{\alpha}
\newcommand{\colcluster}{\mathtt{c}}
\newcommand{\rowcluster}{\mathtt{r}}
\newcommand{\icedomain}{\mathcal{D}}
\newcommand{\basalfriction}{{q}}
\newcommand{\normalvec}{\nu}
\newcommand{\searchdir}{{\widehat{\basalfriction}}}
\newcommand{\preconditioner}{\widetilde{H}}
\newcommand{\candidatepts}{X}
\newcommand{\eigenvectormatrix}{P}
\newcommand{\candidatepoint}{x}
\newcommand{\velocity}{v}
\newcommand{\pressure}{p}
\newcommand{\rbfweight}{c}
\newcommand{\stress}{\sigma}
\newcommand{\tangentop}{T}
\newcommand{\strain}{\varepsilon}
\newcommand{\bodyforce}{f}
\newcommand{\observations}{y}
\newcommand{\rbf}{\varphi}
\newcommand{\fepoint}{p}
\newcommand{\stokesrobincoeff}{{s}}
\newcommand{\regrobincoeff}{{s}}
\newcommand{\regrhs}{{f}}

\newcommand*{\vertbar}{\rule[-1ex]{0.5pt}{2.5ex}}
\newcommand*{\horzbar}{\rule[.5ex]{2.5ex}{0.5pt}}

\newcommand{\genericvec}{\bm{x}}
\newcommand{\concentration}{c} % NOTE: CHECK for collisions!!!
\newcommand{\advdomain}{\Omega}

\usepackage{amsopn}
\DeclareMathOperator{\diag}{diag}
\DeclareMathOperator{\interpolate}{Interpolate}
\DeclareMathOperator{\Span}{span}
\DeclareMathOperator{\Var}{Var}
\DeclareMathOperator{\vol}{vol}
\DeclareMathOperator{\avg}{avg}
\DeclareMathOperator*{\argmax}{arg\,max}
\DeclareMathOperator*{\argmin}{arg\,min}
\DeclareMathOperator{\dist}{dist}
\DeclareMathOperator{\diam}{diam}
\DeclareMathOperator{\nbrs}{nbrs}

\newcommand{\computediracresponse}[1]{\text{compute\_dirac\_comb\_response}\left( #1 \right)}
\newcommand{\ellipsoidsintersect}[2]{\text{ellipsoids\_intersect}\left( #1, #2 \right)}
\newcommand{\choosebatch}[1]{\text{choose\_sample\_point\_batch}\left( #1 \right)}
\newcommand{\computekernelentries}[2]{\text{compute\_approximate\_kernel\_entries}\left( #1 , #2 \right)}
\newcommand{\computeweightingentries}[1]{\text{compute\_weighting\_function\_entries}\left( #1 \right)}

\DeclareFontFamily{U}{wncy}{}
\DeclareFontShape{U}{wncy}{m}{n}{<->wncyr10}{}
\DeclareSymbolFont{mcy}{U}{wncy}{m}{n}
\DeclareMathSymbol{\Sh}{\mathord}{mcy}{"58} 

\SetEndCharOfAlgoLine{}
%\SetKwIF{If}{ElseIf}{Else}{if}{}{else if}{else}{end if}

%\renewcommand{\algorithmicrequire}{\textbf{Input:}}
%\renewcommand{\algorithmicensure}{\textbf{Output:}}

\newcommand{\iFF}{\mathcal{F}}
\newcommand{\D}{\mathcal{D}}
\newcommand{\R}{\mathbb{R}}
\newcommand{\ipar}{\basalfriction} % make parameters consistent across problems

\makeatletter
\let\ftype@table\ftype@figure
\makeatother

%%% Local Variables: 
%%% mode:latex
%%% TeX-master: "ex_article"
%%% End: 
